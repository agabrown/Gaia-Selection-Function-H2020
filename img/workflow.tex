%%%%%%%%%%%%%%%%%%%%%%%%%%%%%%%%%%%%%%%%%%%%%%%%%%%%%%%%%%%%%%%%%%%%%%%%%%%%%%%%%%%%%%%%%%
%
% TeX file workflow.tex
% Uses: tikz package
% Last Updated:  2020.02.27
% First Created: 2020.02.25
%
% Title: CoG workflow.
%
% Description: Graphic that shows approach to the work plan for CoG.
%
%%%%%%%%%%%%%%%%%%%%%%%%%%%%%%%%%%%%%%%%%%%%%%%%%%%%%%%%%%%%%%%%%%%%%%%%%%%%%%%%%%%%%%%%%%

\documentclass{article}

\usepackage{times,layouts}
\usepackage{tikz,html,amsmath}
\usetikzlibrary{arrows,shapes,calc}
\usetikzlibrary{backgrounds}

\usepackage[paperwidth=145mm,paperheight=80mm,
left=-2mm,top=3mm,bottom=0mm,right=0mm,
noheadfoot,marginparwidth=0pt,includemp=false,
textwidth=140mm,textheight=75mm]{geometry}

\newcommand\showpage{%
\setlayoutscale{0.5}\setlabelfont{\tiny}\printheadingsfalse\printparametersfalse
\currentpage\pagedesign}

\hypersetup{pdftitle={CoG workflow},
            pdfsubject={Graphic that shows approach to the work plan for CoG},
            pdfauthor=Anthony Brown}

\xdefinecolor{mptab10blue}{rgb}{0.12156862745098039, 0.4666666666666667, 0.7058823529411765}
\xdefinecolor{mptab10orange}{rgb}{1.0, 0.4980392156862745, 0.054901960784313725}
\xdefinecolor{mptab10green}{rgb}{0.17254901960784313, 0.6274509803921569, 0.17254901960784313}
\xdefinecolor{mptab10red}{rgb}{0.8392156862745098, 0.15294117647058825, 0.1568627450980392}
\xdefinecolor{mptab10purple}{rgb}{0.5803921568627451, 0.403921568627451, 0.7411764705882353}
\xdefinecolor{mptab10brown}{rgb}{0.5490196078431373, 0.33725490196078434, 0.29411764705882354}
\xdefinecolor{mptab10magenta}{rgb}{0.8901960784313725, 0.4666666666666667, 0.7607843137254902}
\xdefinecolor{mptab10grey}{rgb}{0.4980392156862745, 0.4980392156862745, 0.4980392156862745}
\xdefinecolor{mptab10olive}{rgb}{0.7372549019607844, 0.7411764705882353, 0.13333333333333333}
\xdefinecolor{mptab10cyan}{rgb}{0.09019607843137255, 0.7450980392156863, 0.8117647058823529}

\tikzstyle{wp}=[rectangle, draw=none, thick, minimum height=5mm, inner
sep=3pt, text badly centered, text width=28mm, fill=mptab10green!60!white]

\tikzstyle{usertest}=[ellipse, draw=none, thick, minimum height=8mm, inner
sep=3pt, text badly centered, text width=14mm, rounded corners=3pt, fill=black!30!white]

\tikzstyle{prototype}=[ellipse, draw=none, thick, minimum height=8mm, inner
sep=3pt, text badly centered, text width=14mm, rounded corners=3pt, fill=mptab10orange]

\tikzstyle{feedback}=[ellipse, draw=none, thick, minimum height=8mm, inner
sep=3pt, text badly centered, text width=14mm, rounded corners=3pt, fill=mptab10blue!60!white]

% Define data flow arrow style
\tikzstyle{workflow}=[->, shorten >=1pt, shorten <=1pt, >=stealth', thick, color=black]

\begin{document}
\pgfdeclarelayer{background}
\pgfsetlayers{background,main}
\begin{tikzpicture}[font=\scriptsize, thick, node distance=0.7cm]
  \node[wp] (analysis1) at (0,0) {Research \& Development};
  \node[wp, below of=analysis1] (spec1) {Implementation requirements};
  \node[wp, below of=spec1] (implement1) {Implementation};
  \node[prototype, right of=spec1, xshift=2.7cm] (proto1) {Prototype v1};
  \node[usertest, below of=proto1, yshift=-0.7cm] (test1) {User testing at community workshop};
  \node[feedback, below of=test1, yshift=-0.7cm] (feedback1) {Feedback};
  \node[below of=implement1] {Phase 1};

  \node[wp, right of=feedback1, xshift=2.7cm] (spec2) {Implementation requirements};
  \node[wp, above of=spec2] (analysis2) {Research \& Development};
  \node[wp, below of=spec2] (implement2) {Implementation};
  \node[prototype, right of=spec2, xshift=2.7cm] (proto2) {Prototype v2};
  \node[usertest, below of=proto2, yshift=-0.7cm] (test2) {User testing at community workshop};
  \node[feedback, below of=test2, yshift=-0.7cm] (feedback2) {Feedback};
  \node[below of=implement2] {Phase 2};

  \draw[workflow] (analysis1.0) -- (proto1);
  \draw[workflow] (spec1.0) -- (proto1);
  \draw[workflow] (implement1.0) -- (proto1);
  \draw[workflow] (proto1) -- (test1);
  \draw[workflow] (test1) -- (feedback1);
  \draw[workflow] (feedback1) -- (analysis2.180);
  \draw[workflow] (feedback1) -- (spec2.180);
  \draw[workflow] (feedback1) -- (implement2.180);

  \draw[workflow] (analysis2.0) -- (proto2);
  \draw[workflow] (spec2.0) -- (proto2);
  \draw[workflow] (implement2.0) -- (proto2);
  \draw[workflow] (proto2) -- (test2);
  \draw[workflow] (test2) -- (feedback2);
  \draw[workflow, dashed] (feedback2) -- ($(feedback2.0)+(1,0)$);
\end{tikzpicture}
\end{document}
