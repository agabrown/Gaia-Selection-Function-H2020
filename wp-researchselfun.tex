\begin{workpackage}{Research and define the Gaia selection function}
  \label{wp:selfungaia}
  \wpstart{6} %Starting Month
  \wpend{\duration} %End Month
  \wptype{RTD} %RTD, DEM, MGT, or OTHER
  \wplead{UCAM}
  \personmonths{ULEI}{2}
  \personmonths{MPG}{6}
  \personmonths{INAF}{10}
  \personmonths*{UCAM}{23}
  \personmonths{NYU}{0}
  \personmonths{MONA}{0}
  
  \makewptable % Work package summary table

  % Work Package Objectives
  \begin{wpobjectives}
    The objective of this work package is to research and develop in detail the description and modelling of the Gaia survey selection function.
    \begin{enumerate}
      \item Develop the overall survey selection function and using that as a starting point create more specialized selection functions, focusing for example the Gaia astrometric, photometric, and spectroscopic surveys and combinations thereof. In addition a selection function for specific subsets of the Gaia survey will be developed, such as binary stars and exoplanets, variable stars, solar system objects, extragalactic sources. It will be essential here to agree to the scope of the work early on, where the development of further specialized Gaia selection function is left to the community who can make use of the tools developed in this work package. 
      \item Investigate to what extent detailed information is needed on the Gaia pointing history, its on-board detection algorithm, and the data losses introduced along the various steps from on-board measurement to the final data products. Can the selection function be reverse engineered from publicly available information?
      \item Interface with DPAC to gain a deeper understanding of the many ingredients of the Gaia selection function, in particular of aspects where public information is not (yet) available. \memo{This should be formulated more clearly.}
    \end{enumerate}
  \end{wpobjectives}

  % Work Package Description
  \begin{wpdescription}
    % Divide work package into multiple tasks.
    % Use \wptask command
    % syntax: \wptask{leader}{contributors}{start-m}{end-m}{title}{description}   

    Description of work carried out in WP, broken down into tasks, and
    with role of partners list. Use the \texttt{\textbackslash wptask} command.

    \wptask{UCAM}{UCAM}{1}{12}{Test}{
      \label{task:wp3test}
      Here we will test the WP Task code. 
    }
    \wptask{ULEI}{All other}{6}{9}{Integrate}{
      \label{task:wp3integrate}
      In this task UZH will integrate the work done in ~\ref{task:wp3test}.
    }    
    \wptask{ULEI}{All other}{9}{12}{Apply}{
      Here all the WP participants will apply the results to...
    }

    \paragraph{Role of partners}
    \begin{description}
      \item[Participant short name] will lead Task~\ref{task:wp3integrate}.
      \item[UoC] will..
    \end{description}
  \end{wpdescription}

  % Work Package Deliverable
  \begin{wpdeliverables}
    % \wpdeliverable[date]{R}{PU}{A report on \ldots}

    \wpdeliverable[36]{UCAM}{R}{PU}{Report on the definition of the model
    specifications.}\label{dev:wp2specs}

    \wpdeliverable[12]{ULEI}{R}{PU}{Report on Feasibility study for the model
    implementation.}\label{dev:wp2implementation}

    \wpdeliverable[24]{ULEI}{R}{PU}{Prototype of model
    implementation.}\label{dev:wp2prototype}

  \end{wpdeliverables}

\end{workpackage}


%%% Local Variables:
%%% mode: latex
%%% TeX-master: "proposal-main"
%%% End:
