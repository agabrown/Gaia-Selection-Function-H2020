%%% Important. To have correct table numberings
\renewcommand{\thetable}{\thesection\alph{table}}

\chapter[Excellence]{Excellence}
\label{cha:excellence}
\instructions{
Your proposal must address a work programme topic for this call for proposals. \\
\textit{This section of your proposal will be assessed only to the extent that it is relevant to that topic.}\\
}

\section{Objectives}
\label{sec:objectives}
\instructions{
Describe the overall and specific objectives for the project1, which should be clear, measurable,
realistic and achievable within the duration of the project. Objectives should be consistent with
the expected exploitation and impact of the project (see section 2).
}

The European Space Agency's Gaia Mission \cite{2016A&A...595A...1G} is the most successful of ESA's space astronomy
missions as measured by the publication rate of papers using its data in one form or another. Over 2500 papers appeared
between April 2018 and February 2020 that use data from the second Gaia data release\footnote{As indicated for example by the \href{https://ui.adsabs.harvard.edu/search/q=citations(bibcode\%3A2018A\%26A...616A...1G)&sort=date\%20desc\%2C\%20bibcode\%20desc&p_=0}{citations to the Gaia DR2 release paper}}. Among the many high profile science
results published by European groups are the discovery of a spiral in the phase space structure of the Milky Way's disk
(\cite{2018Natur.561..360A} indicative of a recent disturbance, most likely due to the Sagittarius dwarf galaxy), the
first direct observational evidence of crystallization in the interiors of white dwarfs \cite{2019Natur.565..202T} (confirming a fifty year old prediction), and the uncovering of a major event 10 billion years into the Milky Way's past
when a collision with a dwarf galaxy took place which contributed to the build-up of the Milky Way's stellar halo and
thick disk \cite{2018Natur.563...85H, 2018MNRAS.478..611B, 2019arXiv190904679B}. 

Many more exciting results have been reported and a common theme is that the astronomy community has so far mostly
focused (understandably) on the `low hanging fruit' in the Gaia data archive. This is manifest in the often drastic cuts
that are done on the quality of the Gaia data in order to obtain samples of stars for which the data analysis is
straightforward (avoiding for example the well known problems associated with the estimation of distance as the inverse
of the parallax \cite{2018A&A...616A...9L}), and in the ignoring of the properties of the Gaia survey itself, in
particular its completeness and source selection strategy. This means that many high-profile science cases that Gaia was
built for cannot be realized because the selection function of the Gaia survey combined with the data cuts leads to
unrepresentative samples from which, for example, the size scale of the structural components of the Milky Way cannot
reliably be derived.

The Gaia selection function is a highly non-trivial combination of the on-board selection of sources, telemetry
losses, selections on data quality during the data processing, and selections on quality before the publication of the
processed data in the Gaia archive \memo{(show examples in the scientific motivation section below)}. As a result, the
selection function for Gaia has only been characterized in highly simplified ways \memo{(provide literature references
in the motivation section below)} even though it will be crucial for a full exploitation of the intrinsic science
potential of the Gaia data, in particular also in view of the legacy value of the Gaia archive which will remain
the standard in fundamental astrophysical data for decades to come.

The objective of this proposal is to build a detailed selection function for the Gaia data and to provide this to the
scientific community in the form of numerical tables and source code that allow users to apply the selection function to
their scientific analyses and thus maximize the outcome thereof. In detail the objectives are:

\begin{enumerate}
    \item Develop a detailed mathematical formulation of a survey selection function, where the goal is to keep this
        as general as possible even if in this project the focus is on the Gaia survey. The formulation should account
        for the layering of user imposed sample selections on top of the survey selection function.
    \item Provide a detailed practical implementation of the Gaia selection function in the form of auxiliary data,
        which will be accessible through the ESA Science Data Centre, and open source tools which will be made available through
        code hosting websites. 
    \item Develop tools to incorporate the selection function in scientific analyses. These tools should allow for
        combining the Gaia survey selection function with user imposed sample restrictions. The tools will be made
        available as open source code through code hosting websites.
    \item Apply the selection function tools to example science cases. This will serve to demonstrate the benefits of
        carefully accounting for the selection function and at the same time provide worked examples to the community of
        prospective users of these tools.
\end{enumerate}

\subsection{Context}
\label{sec:context}

ESA's Gaia mission \cite{2016A&A...595A...1G} represents a European breakthrough in astrophysics, a cornerstone
mission which was launched in 2013 aimed at producing the most accurate 3D map of the Milky Way to date. The resulting
stereoscopic census of our Galaxy represents a giant leap in astrometric accuracy (reaching the 10--20 micro-arcsecond
regime, which is equivalent to knowing the 3D positions of stars to 10 per cent precision over distances as large as $20\,000$ to $30\,000$ lightyears) complemented by the only full sky homogeneous photometric survey with an angular resolution comparable to that
of the Hubble Space Telescope, as well as the largest spectroscopic survey ever undertaken. 

The primary scientific aim of the mission is to map the structure of our Galaxy and unravel its formation history and
subsequent evolution. Current cosmological models envisage the formation of large galaxies through the merging of
smaller structures. Deciphering the assembly history of our Galaxy requires a detailed mapping of the structure,
dynamics, chemical composition, and age distribution of its stellar populations. Ideally one would like to `tag'
individual stars to each of the progenitor building blocks of the Galaxy \cite{2002ARA&A..40..487F}. The Gaia mission is
designed to provide the required fundamental data in the form of distances (through parallaxes), space velocities
(through proper motions and radial velocities) and astrophysical characterisation (through multi-colour photometry) for
massive numbers of stars throughout most of the Galaxy. It should be stressed however that Gaia is not simply a `Milky
Way mission' but is truly a multi-faceted {\em astrophysics mission} which will provide exciting scientific results
covering many topics including: fundamental stellar physics across the Hertzsprung-Russell diagram, the characterisation of
tens of millions of binary stars, unique samples of variable stars of nearly all types (including key cosmological
distance calibrators), detection and orbital classification of thousands of extra-solar planetary systems, a
comprehensive survey of objects ranging from huge numbers of minor bodies in our Solar System, through galaxies in the
nearby Universe, to over a million distant quasars. Gaia will also provide a number of stringent tests of general
relativity. Last but not least, a massive survey such as Gaia will uncover many surprises that the Universe still holds
in store for us.

The data processing for the Gaia mission is carried out by a European consortium of institutes and funding agencies, the Data Processing and Analysis Consortium (DPAC). The task of the DPAC is to turn the raw telemetry from Gaia into data products ready for use by astronomers and scientists around the world. The data products are both the basic astrometric, photometric, and spectroscopic results as well as advanced data products, such as the astrophysical characterization of stars, derived from the former. Through an independent unit within the consortium the DPAC also ensures the quality of the published data through extensive technical and scientific validation. The publicly released data, which are disseminated through the ESA Science Data Center and its partners are accompanied by extensive documentation which is also produced by DPAC. It should be stressed here that the DPAC holds no proprietary rights to the data which means that once the data products are validated and documented they are released publicly and world-wide without delay. In this sense Gaia represents a uniquely open survey concept.

From the early days of the Gaia mission it was foreseen to have incremental data releases based on increasing amounts of collected telemetry. The various data products would be released in stages (where, for example, some of the advanced data products such as exoplanet catalogues can only be released after sufficient data has been collected and the various astrometric calibrations have reached the required levels of precision) and with increasing accuracy for each release. To date two data releases have taken place, Gaia DR1 in September 2016 and Gaia DR2 in April 2018. Both releases have been highly successful and have had major impacts on the fields of Galactic archaeology; Milky Way mass estimation; streams, dwarf galaxies, globular clusters; open clusters; white dwarfs; the characterization of exoplanet host stars and determining the absolute sizes of protoplanetary disks, to name a number of prominent examples. The studies of individual stars and stellar systems benefit immensely from the now easily available high precision parallax data.

The next releases foreseen are Gaia EDR3 in the third quarter of 2020 and Gaia DR3 in the second half of 2021. The following release (Gaia DR4) will be based on all data collected during the nominal mission lifetime of Gaia and feature the full suite of data products, including the epoch data (individual measurements) for the astrometry, photometry, and spectroscopy\footnote{See the \href{https://www.cosmos.esa.int/web/gaia/release}{Gaia data release scenario} pages.}. These future Gaia data releases are expected to have an even larger impact on astronomy due to the much richer set of data products allowing for the combination of high accuracy astrometry and radial velocities with detailed astrophysical characterization of the source in the Gaia catalogue (stellar parameters including chemical compositions and variability, parameters of multiple stars and exoplanets, medium resolution spectra around the Ca triplet region for millions of stars to magnitude 13, low resolution prism spectrophotometry for all sources in the Gaia catalogue). 

Despite the major successes of Gaia DR1 and Gaia DR2 there is one serious weakness of the releases is that the DPAC does not provide a detailed selection function for the Gaia survey. This aspect has never been included in the funded DPAC efforts and will not be added as DPAC work in the future. This implies that major Gaia science cases cannot be addressed by the astronomical community to the full potential of the quality of the data. In the next section we motivate why a detailed selection function is needed and how this will enhance the scientific exploitation of the data from this flagship European space mission.

\paragraph{Gaia instruments and survey strategy} For the understanding of the work proposed here it is useful to briefly summarize the characteristics of the Gaia mission and how the measurement are collected and subsequently processed on ground. Much more detail can be found in the paper describing the Gaia mission \cite{2016A&A...595A...1G}. The Gaia spacecraft and its instruments were designed to collect data that will allow the determination of highly accurate positions, parallaxes, and proper motions for $>1$ billion sources brighter than magnitude $20.7$ in its white-light photometric band $G$ (covering the range $330$--$1050$~nm). The astrometry is complemented by multi-colour photometry, measured for all sources observed by Gaia, and radial velocities which are collected for stars brighter than $G\approx17$. Gaia carries two telescopes of which the collected light is combined into a single focal plane servicing the three main instruments. The astrometric instrument collects source images in the $G$-band, where the fundamental inputs to the astrometric data processing consist of the precise times when the image centroids pass a fiducial line in the focal plane of the instrument. The photometric instrument is realised through two prisms dispersing the light entering the field of view of two dedicated sets of CCDs. The Blue Photometer (BP) operates over the wavelength range $330$--$680$ nm, while the Red Photometer (RP) covers the wavelength range $640$--$1050$ nm. The data collected by the photometric instrument consists of low resolution spectrophotometric measurements of the source spectral energy distributions. The spectroscopic instrument, also called the radial-velocity spectrometer (RVS), collects medium resolution ($R\sim11\,700$) spectra over the wavelength range $845$--$872$~nm, centred on the Calcium triplet region. The spectra are collected for all sources to $G\approx17$ (16$^\text{th}$ magnitude in the RVS filter band).

The Gaia sky survey strategy is derived from that of the Hipparcos mission and and relies on the spacecraft slowly spinning around the axis perpendicular to the lines of sight of the two telescopes. Every six hours the Gaia telescopes can a great circle on the sky with an across scan field of view size of $0.7^\circ$. By slowly precessing the spin axis around the direction to the Sun the full sky can be surveyed with optimal sky coverage uniformity over the five years nominal mission lifetime. 

The processing of the observations collected by Gaia is carried out by the DPAC through a series of complex pipelines which are part of a large iterative system\citep[see section 7 in][]{2016A&A...595A...1G}.

Although the Gaia survey selection function is seemingly very simple, with only a magnitude limit imposed on the observations, each of the above components of the Gaia mission adds complexity to the survey selection function. A few examples are listed here:
\begin{itemize}
    \item The data collection on the spacecraft is limited by the amount of sources that can be handled at any on time, meaning that for dense regions on the sky (above $10^6$, $750\,000$, and $35\,000$ sources~deg$^{-2}$ for the astrometric, photometric, and spectroscopic instruments respectively) not all sources can be measured leading brighter effective survey limits in those regions. In addition the on-board storage capacity limitations combined with available ground station time lead to deletion of data on occasions when both Gaia telescopes are scanning along the Galactic plane for prolonged periods of time.
    \item Despite the optimal sky survey strategy realized through the revolving scanning technique, the number of times any given source is observed does vary with celestial position, in particular with the ecliptic latitude.
    \item Along each the various on-ground data processing steps decisions are taken on which data and/or sources to process, where periods of "bad inputs" are excluded. This leads to gaps in the time coverage of measurements for a given source or to effectively brighter survey limits covering certain time periods (which translates to certain areas on the sky). Additionally, in early data releases the processing for certain data products may be limited to brighter sources or sources with specific astrophysical characteristics.
\end{itemize}
Each of the above examples complicates the Gaia survey selection function which in the end is very non-trivial to describe and handle. Nevertheless we must make an effort to provide the best possible description of the selection function as well as tools to incorporate it into scientific analyses as we will motivate next.

\subsection{Scientific motivation}
\label{sec:scientific-motivation}

\textbf{Douglas: Apologies, I haven't found time to properly attempt this yet.}

A selection function quantifies how the objects in an astronomical catalogue were chosen from the countless asteroids, planets, stars and galaxies in the Universe. Without the selection function, we can only make state

Without a selection function, it is impossible to extrapolate \textit{statements about the objects in a catalogue} into \textit{insights in fundamental physics}. 

The selection function is necessary in order to draw conclusions about fundamental physics from the limited and biased fraction of objects in our catalogues. Without a selection function, we could only make statements about the objects in our catalogues.

The selection function can be calculated for any hypothetical object, returning one if that object would have been included in the catalogue and zero if not. The selection function can also take values between zero and one, in which case we interpret the selection function as giving the probability that the object would appear in the catalogue.

The Gaia selection function is hard.



Detailed motivation of the need for a selection, use a few example science cases to make the point.

\begin{itemize}
    \item What prominent Gaia science cases cannot (ever?) be done without a well-defined and computationally tractable
        selection function
        \begin{itemize}
            \item Structural parameters of the Milky Way (disk scale length/height etc)
            \item Dark matter sub-halo finding through gaps in streams
            \item Binary population parameters (frequency, parameter distributions), but also stellar census (mass function of single stars)
            \item Exoplanet population parameters (frequency, parameter distributions)
            \item Local dark matter density
        \end{itemize}
    \item What is the selection function.
        \begin{itemize}
            \item Needs to be clearly defined and will probably require someone to work on the detailed definition and
                mathematical formulation (if the latter has not already been done).
            \item How does it relate to survey completeness; do we include completeness in the definition.
            \item Point out that multiple layers of selection may take place in any scientific investigation of Gaia or
                other astronomy data. The formulation/definition of the selection function should allow for layers of
                selection.
            \item To keep the work manageable in this project we will only provide the `basic' Gaia selection function.
                Accounting for the effect of additional user-imposed selection of data will be through tools also
                developed in this project.
        \end{itemize}
    \item Impact on the rest of astrophysics
        \begin{itemize}
            \item The Gaia catalogue is now being used to define target selection for other surveys (e.g. 4MOST). The completeness of these derivative surveys will be defined by the completeness of Gaia.
            \item We will implement selection functions for other surveys (e.g. APOGEE, LAMOST) in the open source software that we produce, that will enable users to, for example, calculate the odds of a star having a parallax in Gaia and a radial velocity in APOGEE.
        \end{itemize}
\end{itemize}

\section{Relation to the work programme}
\label{sec:relation-to-work-programme}
\instructions{
Indicate the work programme topic to which your proposal relates, and explain how your proposal
addresses the specific challenge and scope of that topic, as set out in the work programme.
}

This proposal is in response to the call ``Space 2018--2020'' and is specifically focused on the topic ``Scientific Data Exploitation'' (SPACE-30-SCI-2020). The work proposed here addresses the SPACE-30-SCI-2020 challenges as follows:
\begin{itemize}
    \item Researching, developing, implementing and publicly providing a detailed Gaia survey selection function will (as motivated above) very much enhance the scientific data exploitation of a flagship European (ESA) space mission. In particular the data from the Gaia mission will have a legacy value for many decades to come. The future scientific exploitation of the Gaia legacy archive will also benefit tremendously from a readily available detailed selection function.
    \item The development of the Gaia selection function will, among others, involve a comparison to or combination with data from other surveys, both ground and space based. \memo{Refer to the relevant methods and WP descriptions} This will enhance our insights into the Gaia selection function but also provide new insights into the selection functions of the other surveys. 
    \item The expertise built up in constructing the Gaia selection function can be transferred to other surveys and thus enhance the science exploitation of future European space missions, or even inform the design of such missions to ensure the resulting surveys have tractable selection functions.
    \item The work proposed here will lead to data products that can be integrated into the ESA Gaia archive and to open source software tools which will be made available through code hosting websites. The combination of data and code will enable the users to include their scientific analyses the Gaia selection function as well as selection functions for Gaia combined with other surveys.
    \item The participants in this proposal represent an international collaboration with in particular the involvement of partners from the USA and Australia, both countries which are active in space exploration and space science.
\end{itemize}

\section{Concept and methodology}
\label{sec:conceptandmethods}
\subsection{(a) Concept}
\label{sec:concept}
\instructions{
    \begin{itemize}
        \item Describe and explain the overall concept underpinning the project. Describe the main
            ideas, models or assumptions involved. Identify any inter-disciplinary considerations and,
            where relevant, use of stakeholder knowledge. Where relevant, include measures taken for
            public/societal engagement on issues related to the project. Describe the positioning of the
            project e.g. where it is situated in the spectrum from ‘idea to application’, or from ‘lab to
            market’. Refer to Technology Readiness Levels where relevant. (See General Annex G of the work
            program);
        \item Describe any national or international research and innovation activities which will be
            linked with the project, especially where the outputs from these will feed into the project;
    \end{itemize}
}

Develop and implement the Gaia selection through research to be conducted by four teams \memo{(co-I plus 1 or 2
postdocs, we'll see what can be financed)} and supported with the expertise form the US/Australian partners.

\subsection{(b) Methodology}
\label{sec:methods}
\instructions{
    \begin{itemize}
        \item Describe and explain the overall methodology, distinguishing, as appropriate, activities
            indicated in the relevant section of the work programme, e.g. for research, demonstration,
            piloting, first market replication, etc.
    \end{itemize}
}

\memo{The tasks below will be mapped to the work packages for this proposal. The WP descriptions will provide more
detail on the work to be done.}

\begin{enumerate}
    \item Operational definition selection/completeness function
        \begin{itemize}
            \item For arbitrary sky survey? More specific to Gaia?
            \item Mathematical formulation; account for multiple layers of selection (spacecraft, data processing, user)
            \item Handling unknown aspects of the selection function (for example information lost along with dropped
                spacecraft telemetry)
        \end{itemize}
    \item Practical realization of Gaia selection function
        \begin{itemize}
            \item Gather the necessary information (on-board detection/selection, filtering during data processing,
                statistics on Gaia sources, …)
            \item Collect data from other surveys
            \item Study spatial resolution limitations
            \item Translate basic selection steps to effect on source astrophysical parameter distributions (stellar
                parameters, binary parameters, population stats, …)
            \item Define a “baseline” selection function as starting point for adding the effect of user defined
                selection steps
        \end{itemize}
    \item Develop tools (code+data) to incorporate selection function in scientific analyses
        \begin{itemize}
            \item Layer user choices onto baseline selection function(s)
            \item How to make this available
            \item Documentation
        \end{itemize}
    \item Apply the selection function tools to selected science cases.
\end{enumerate}

\section{Ambition}
\label{sec:ambition}
\instructions{
    \begin{itemize}
        \item Describe the advance your proposal would provide beyond the state-of-the-art, and the
            extent the proposed work is ambitious.
        \item Describe the innovation potential (e.g. ground-breaking objectives, novel concepts and
            approaches, new products, services or business and organisational models) which the proposal
            represents. Where relevant, refer to products and services already available on the market.
            Please refer to the results of any patent search carried out.
    \end{itemize}
}

\begin{itemize}
    \item Currently no selection function is available for Gaia catalogue, only ad-hoc implementations for specific
        science cases.
        \begin{itemize}
            \item Most works in the end ignore selection function effects.
        \end{itemize}
    \item Readily available selection function will enhance all scientific analyses of Gaia data and lead to much better
        reproducibility of results.
    \item This project will result in clearer definition of selection function concept, which can be applied to other
        surveys too.
    \item Expertise gained in this project can be transferred to other surveys.
\end{itemize}

%%% Local Variables:
%%% mode: latex
%%% TeX-master: "proposal-main"
%%% End:
