\chapter{Implementation}
\label{cha:implementation}

\section{Work plan --- Work packages, deliverables}
\label{sec:work-plan}

\subsection{Overall structure}
\label{sec:wpstructure}

The overall approach to the work plan is to have the research and development of the selection function progress in parallel with the implementation thereof. We made this choice because of the complexities of the selection function and its realization in the form of tools and auxiliary data, and because of the relatively short duration of the project. The work flow we foresee for {\acro} is shown in \figref{fig:workflow}. It is an iterative workflow in which the findings in each of work packages \ref{wp:selfungaia}--\ref{wp:scienceappl} will inform the others. For example the research on the selection function will lead to requirements on the implementation, while at the same time the conversion of research ideas into practical tools will lead to the questioning and revision of those same ideas. At several points during the project the results will be combined into prototype implementations of the selection function tools. These will be tested both internally to the project and through the public availability of one prototype implementation. The results from the tests will feed into the next round of research, development, and implementation. The scientific nature of much of the work conducted in {\acro} will very much benefit from an iterative approach. Close coordination between the work packages will of course be essential and this is foreseen in the management structure.

\begin{figure}
    \centering
    \includegraphics[width=\linewidth]{workflow.pdf}
    \caption{Iterative workflow for CoG.}
    \label{fig:workflow}
\end{figure}

\begin{figure}[ht!]
    \centering
    \includegraphics[width=\linewidth]{cog-gantt.pdf}
    \caption{Gantt chart for the {\acro} project.}
    \label{fig:gantt}
\end{figure}

\subsection{Timing of the work packages}
\label{sec:wptiming}

The overall work plan is structured as follows:
\begin{itemize}
    \item The first 6 months will be used to recruit the researchers to be funded from this proposal and to carry out the tasks in work package \ref{wp:selfundefinition}. Having a mathematical framework for the selection function in place early on in the project is important for guiding the research and the implementation phases.
    \item During the next three years of the project the efforts in work packages \ref{wp:selfungaia} to \ref{wp:scienceappl} will happen mostly in parallel for the reasons outlined above. The iterative approach outline in \secref{sec:wpstructure} will be used in this phase.
 \end{itemize}

\makewplist

%\subsection{Work package description}
%\label{sec:wps}

\tablecaption{Description of work packages}
\begin{supertabular}{p{\textwidth}}
    \omit \tabularnewline
\end{supertabular}
%%%%%%%%%%%%%%%%%%%%%%%%%%%%%%
%  Work Package Description  %
%%%%%%%%%%%%%%%%%%%%%%%%%%%%%%

\begin{workpackage}{Management}
  \label{wp:management} %change and use appropriate description

  %%%%%%%%%%%%%%%%%% TOP TABLE %%%%%%%%%%%%%%%%%%%%%%%%%%%%%
  % Data for the top table
  \wpstart{1} %Starting Month
  \wpend{36} %End Month
  \wptype{Activity type} %RTD, DEM, MGT, or OTHER
  \wplead{ULEI}

  % Person Months per participant (required, max 7, * for leader)  
  % syntax: \personmonths{Participant number}{value}    (not wp leader)
  %     or  \personmonths{Participant short name}{value} (not wp leader)
  %         \personmonths*{Participant number}{value}    (wp leader)
  % for example:
  \personmonths*{ULEI}{12}
  \personmonths{MPIA}{3}
  \personmonths{INAF}{2}
  % etc.

  \makewptable % Work package summary table

  % Work Package Objectives
  \begin{wpobjectives}
    This work package has the following objectives:
    \begin{enumerate}
      \item To develop ....
      \item To apply this ....
      \item etc.
    \end{enumerate}
  \end{wpobjectives}

  % Work Package Description
  \begin{wpdescription}
    % Divide work package into multiple tasks.
    % Use \wptask command
    % syntax: \wptask{leader}{contributors}{start-m}{end-m}{title}{description}   

    Description of work carried out in WP, broken down into tasks, and
    with role of partners list. Use the \texttt{\textbackslash wptask} command.

    \wptask{ULEI}{ULEI}{1}{12}{Test}{
      \label{task:wp1test}
      Here we will test the WP Task code. 
    }
    \wptask{ULEI}{All other}{6}{9}{Integrate}{
      \label{task:wp1integrate}
      In this task UZH will integrate the work done in ~\ref{task:wp1test}.
    }    
    \wptask{ULEI}{All other}{9}{12}{Apply}{
      Here all the WP participants will apply the results to...
    }

    \paragraph{Role of partners}
    \begin{description}
      \item[Participant short name] will lead Task~\ref{task:wp1integrate}.
      \item[UoC] will..
    \end{description}
  \end{wpdescription}

  % Work Package Deliverable
  \begin{wpdeliverables}
    % Data for the deliverables and milestones  tables
    % syntax: \deliverable[delivery date]{nature}{dissemination
    % level}{description} 
    %
    % nature: R = Report, DEM = Demonstrator, DEC = Websites, media, etc, OTHER = Other
    % dissemination level: PU = Public, CO = Confidential, CI = CLassified.
    % 
    % \wpdeliverable[date]{R}{PU}{A report on \ldots}

    \wpdeliverable[6]{ULEI}{R}{PU}{Data management plan}\label{dev:wp1datamanagement}

    \wpdeliverable[12]{ULEI}{R}{PU}{Report on Feasibility study for the model
    implementation.}\label{dev:wp1implementation}

    \wpdeliverable[24]{ULEI}{R}{PU}{Prototype of model
    implementation.}\label{dev:wp1prototype}

  \end{wpdeliverables}

\end{workpackage}


%%% Local Variables:
%%% mode: latex
%%% TeX-master: "proposal-main"
%%% End:

\begin{workpackage}{Definition of the selection function}
  \label{wp:selfundefinition}
  \wpstart{1} %Starting Month
  \wpend{12} %End Month
  \wptype{RTD} %RTD, DEM, MGT, or OTHER
  \wplead{MPG}
  \personmonths{ULEI}{1}
  \personmonths*{MPG}{2}
  \personmonths{INAF}{1}
  \personmonths{UCAM}{1}
  \personmonths{NYU}{1}
  \personmonths{MONA}{1}
  
  \makewptable % Work package summary table

  % Work Package Objectives
  \begin{wpobjectives}
    This objective of this work package is to research and implement a precise mathematical formulation of the concept of a survey selection function. The results will be written up as a scientific paper (to be published in the peer-reviewed literature) that will guide the rest of the work to be done within {\acro}.  The definition of the selection function will account for the following aspects:
    \begin{itemize}
        \item 
    \end{itemize}
  \end{wpobjectives}

  % Work Package Description
  \begin{wpdescription}
    % Divide work package into multiple tasks.
    % Use \wptask command
    % syntax: \wptask{leader}{contributors}{start-m}{end-m}{title}{description}   

    Description of work carried out in WP, broken down into tasks, and
    with role of partners list. Use the \texttt{\textbackslash wptask} command.

    \wptask{UCAM}{UCAM}{1}{12}{Test}{
      \label{task:wp2test}
      Here we will test the WP Task code. 
    }
    \wptask{ULEI}{All other}{6}{9}{Integrate}{
      \label{task:wp2integrate}
      In this task UZH will integrate the work done in ~\ref{task:wp2test}.
    }    
    \wptask{ULEI}{All other}{9}{12}{Apply}{
      Here all the WP participants will apply the results to...
    }

    \paragraph{Role of partners}
    \begin{description}
      \item[Participant short name] will lead Task~\ref{task:wp2integrate}.
      \item[UoC] will..
    \end{description}
  \end{wpdescription}

  % Work Package Deliverable
  \begin{wpdeliverables}
    % \wpdeliverable[date]{R}{PU}{A report on \ldots}

    \wpdeliverable[36]{UCAM}{R}{PU}{Report on the definition of the model
    specifications.}\label{dev:wp2specs}

    \wpdeliverable[12]{ULEI}{R}{PU}{Report on Feasibility study for the model
    implementation.}\label{dev:wp2implementation}

    \wpdeliverable[24]{ULEI}{R}{PU}{Prototype of model
    implementation.}\label{dev:wp2prototype}

  \end{wpdeliverables}

\end{workpackage}


%%% Local Variables:
%%% mode: latex
%%% TeX-master: "proposal-main"
%%% End:

\begin{workpackage}{Research and develop the Gaia selection function}
  \label{wp:selfungaia}
  \wpstart{7} %Starting Month
  \wpend{\duration} %End Month
  \wptype{RTD} %RTD, DEM, MGT, or OTHER
  \wplead{UCAM}
  \personmonths{ULEI}{2}
  \personmonths{MPIA}{6}
  \personmonths{INAF}{10}
  \personmonths*{UCAM}{23}
  \personmonths{NYU}{0}
  \personmonths{MONA}{0}
  
  \makewptable % Work package summary table

  % Work Package Objectives
  \begin{wpobjectives}
    The objective of this work package is to research and develop in detail the description and modelling of the Gaia survey selection function.
    \begin{enumerate}
      \item Develop the overall survey selection function and using that as a starting point create more specialized selection functions, focusing for example the Gaia astrometric, photometric, and spectroscopic surveys and combinations thereof. In addition a selection function for specific subsets of the Gaia survey will be developed, such as binary stars and exoplanets, variable stars, solar system objects, extragalactic sources. It will be essential here to agree to the scope of the work early on, where the development of further specialized Gaia selection function is left to the community who can make use of the tools developed in this work package. 
      \item Investigate to what extent detailed information is needed on the Gaia pointing history, its on-board detection algorithm, and the data losses introduced along the various steps from on-board measurement to the final data products. Can the selection function be reverse engineered from publicly available information?
      \item Interface with DPAC to gain a deeper understanding of the many ingredients of the Gaia selection function, in particular of aspects where public information is not (yet) available. \memo{This should be formulated more clearly.}
    \end{enumerate}
  \end{wpobjectives}

  % Work Package Description
  \begin{wpdescription}
    % Divide work package into multiple tasks.
    % Use \wptask command
    % syntax: \wptask{leader}{contributors}{start-m}{end-m}{title}{description}   

    \wptask{UCAM}{UCAM}{7}{\duration}{Scientific coordination}{
      \label{task:wp3coordination}
       Scientific coordination of the research and development of the selection function and coordination of writing the paper corresponding to deliverable \ref{dev:wp3GSFfinal}.
    }
    \wptask{UCAM}{All other}{7}{\duration}{R\&D top level selection functions}{
      \label{task:wp3toplevelGSF}
      Research and development of the top-level Gaia selection function which will describe the probability that a source enters the Gaia astrometric, photometric or radial velocity surveys, and combinations thereof.
    }    
    \wptask{MPIA}{All other}{7}{\duration}{R\&D spectroscopic selection functions}{
      \label{task:wp3spectroscopicGSF}
      Research and development of the Gaia spectroscopic selection function. This refers to the probability that a BP/RP and/or and RVS spectrum was measured for a source in the Gaia survey.
    }
    \wptask{INAF}{All other}{7}{\duration}{R\&D specialized selection functions}{
      \label{task:wp3specializedGSF}
      Research and develop the methods needed to create selection functions for specific subsets of the Gaia survey, such as binary stars or extragalactic objects. Apply this to example cases.
    }

    \paragraph{Role of partners}
    \begin{description}
      \item[UCAM] will lead Tasks~\ref{task:wp3coordination} and \ref{task:wp3toplevelGSF}.
      \item[MPIA] will lead Task~\ref{task:wp3toplevelGSF}.
      \item[INAF] will lead Task~\ref{task:wp3specializedGSF}
      \item[All partners] will contribute to the research and development of the Gaia selection functions. 
    \end{description}
  \end{wpdescription}

  % Work Package Deliverable
  \begin{wpdeliverables}
    % \wpdeliverable[date]{R}{PU}{A report on \ldots}
    \wpdeliverable[18]{UCAM}{R}{PU}{Documentation of preliminary top level Gaia selection functions.}\label{dev:wp3version1selfun}
    \wpdeliverable[36]{UCAM}{R}{PU}{Documentation of the Gaia selections functions (also to be submitted to as papers to a peer-reviewed journal)}\label{dev:wp3GSFfinal}
    \memo{Add further deliverables? Such as intermediate reports for each of the tasks above?}
  \end{wpdeliverables}

\end{workpackage}


%%% Local Variables:
%%% mode: latex
%%% TeX-master: "proposal-main"
%%% End:

\begin{workpackage}{Practical implementation and dissemination of the Gaia selection function}
  \label{wp:selfunimplementation}
  \wpstart{7} %Starting Month
  \wpend{\duration} %End Month
  \wptype{RTD} %RTD, DEM, MGT, or OTHER
  \wplead{ULEI}
  \personmonths*{ULEI}{23}
  \personmonths{MPG}{6}
  \personmonths{INAF}{2}
  \personmonths{UCAM}{2}
  \personmonths{NYU}{0}
  \personmonths{MONA}{0}
 
  \makewptable % Work package summary table

  % Work Package Objectives
  \begin{wpobjectives}
    This work package provides the practical implementation of the Gaia selection functions in terms of data and associated source code. Dissemination of these results through the ESA Gaia archive and code hosting web-sites is also part of this work package. The detailed objectives are:
    \begin{enumerate}
      \item Implement the selection function as defined and developed in detail in work packages \ref{wp:selfundefinition} and \ref{wp:selfungaia} in the form of open source code and associated numerical data (in the form of tables or any other convenient format).
      \item Implement a tool that allows for layering user defined selections on the Gaia archive data on top of the survey selection functions.
      \item Provide implementations of a number of combined selection functions for intersections of Gaia and selected other surveys.
      \item Identify code hosting options and make the Gaia selection function source code available publicly.
      \item Agree with ESA/DPAC on the hosting of the numerical data associated with the Gaia selection function in the ESA archive ecosystem.
    \end{enumerate}
  \end{wpobjectives}

  % Work Package Description
  \begin{wpdescription}
    % Divide work package into multiple tasks.
    % Use \wptask command
    % syntax: \wptask{leader}{contributors}{start-m}{end-m}{title}{description}
    \wptask{ULEI}{ULEI}{1}{\duration}{Scientific coordination}{
      \label{task:wp4coordination}
      Scientific coordination of the implementation and dissemination of the Gaia survey selection function. This includes setting up the code hosting and interfacing with ESA/DPAC on hosting the numerical data for the selection function in the Gaia archive.
    }
    \wptask{ULEI}{All other}{7}{\duration}{Implementation}{
      \label{task:wp4implement}
      Implement the Gaia selection functions in terms of open source software tools and associated numerical data. Test the implementation through applications to science cases. Port the implementation to the relevant code hosting site and transfer the numerical data to the ESA Gaia archive.
    }    
    \wptask{MPG}{All other}{7}{\duration}{Implementation of tools to include user selections}{
      \label{task:wp4layers}
      Develop and implement tools to chain together the Gaia survey selection function with additional user imposed selection on the Gaia archive data. The tools should produce an overall effective selection function for the user-selected sample.
    }
    \wptask{INAF}{All other}{7}{\duration}{Implementation of tools for combined selection functions}{
      \label{task:wp4combine}
      Implement the selection functions for combinations of Gaia and other surveys based on the methodology developed in work package \ref{wp:selfuncombine}.
    }

    \paragraph{Role of partners}
    \begin{description}
      \item[ULEI] will lead Task~\ref{task:wp4coordination}.
      \item[MPG] will lead Task~\ref{task:wp4layers}.
      \item[INAF] will lead Task~\ref{task:wp4combine}.
      \item[All partners] will contribute to the implementation and testing of the tools developed in this work package. In particular the science applications pursued by the partners will constitute a strong test of the Gaia selection function implementation.
    \end{description}
  \end{wpdescription}

  % Work Package Deliverable
  \begin{wpdeliverables}
    % \wpdeliverable[date]{R}{PU}{A report on \ldots}
    \wpdeliverable[24]{ULEI}{DEM}{PU}{Prototype of open source and open data implementation of Gaia selection function.}\label{dev:wp4prototype}
    \wpdeliverable[\duration]{ULEI}{DEC}{PU}{Open source and open data implementation of Gaia selection function.}
    \memo{Probably need a few more deliverables as concrete checks on progress.}
  \end{wpdeliverables}

\end{workpackage}


%%% Local Variables:
%%% mode: latex
%%% TeX-master: "proposal-main"
%%% End:

\begin{workpackage}{Selection function for combinations of Gaia and other surveys}
  \label{wp:selfuncombine}
  \wpstart{7} %Starting Month
  \wpend{\duration} %End Month
  \wptype{RTD} %RTD, DEM, MGT, or OTHER
  \wplead{MPG}
  \personmonths{ULEI}{2}
  \personmonths*{MPG}{15}
  \personmonths{INAF}{12}
  \personmonths{UCAM}{2}
  \personmonths{NYU}{0}
  \personmonths{MONA}{0}

  \makewptable % Work package summary table

  % Work Package Objectives
  \begin{wpobjectives}
    The objective of this work package is to research and develop methods to derive selection functions for the combination of Gaia and other large sky surveys. A concrete implementation for a few cases will be implemented in work package \ref{wp:selfunimplementation}. The detailed objectives are
    \begin{enumerate}
      \item Research and develop a generic method for constructing selection functions for combinations of surveys.
      \item Apply this method to a few concrete cases of the combination of Gaia with another survey.
    \end{enumerate}
  \end{wpobjectives}

  % Work Package Description
  \begin{wpdescription}
    % Divide work package into multiple tasks.
    % Use \wptask command
    % syntax: \wptask{leader}{contributors}{start-m}{end-m}{title}{description}   
    \wptask{MPG}{All other}{7}{\duration}{Scientific coordination.}{
      \label{task:wp5coordination}
      Scientific coordination of the research and development of the combined selection function and coordination of writing the paper corresponding to deliverable \ref{dev:wp5finalreport}.
    }
    \wptask{MPG}{All other}{7}{\duration}{Generic combination method.}{
      \label{task:wp5method}
      Research and develop a generic method for constructing selection functions for combinations of surveys.
    }
    \wptask{MPG}{All other}{7}{\duration}{Example combined selection functions.}{
      \label{task:wp5examples}
      For a few selected examples construct the selection function for the combination of Gaia and another survey, using the method developed in task~\ref{task:wp5method}.
    }    

    \paragraph{Role of partners}
    \begin{description}
      \item[MPG] will lead Tasks~\ref{task:wp5coordination}, \ref{task:wp5method}, and \ref{task:wp5examples}.
      \item[All other partners] will contribute to tasks \ref{task:wp5method} and \ref{task:wp5examples}.
    \end{description}
  \end{wpdescription}

  % Work Package Deliverable
  \begin{wpdeliverables}
    % \wpdeliverable[date]{R}{PU}{A report on \ldots}
    \wpdeliverable[\duration]{MPG}{R}{PU}{Report documenting the methods to construct combined selection functions (also to be submitted to peer-reviewed journal).}\label{dev:wp5finalreport}
  \end{wpdeliverables}

\end{workpackage}


%%% Local Variables:
%%% mode: latex
%%% TeX-master: "proposal-main"
%%% End:

\input{wp-scienceappl}

\makedeliverablelist

\subsection{Dependencies}
\label{sec:dependencies}

\figrefcap{fig:dependencies} shows the dependencies between the various work packages in \acro. The arrows indicate a ‘depends on’ relationship. For example WP\ref{wp:selfuncombine} depends on WP\ref{wp:selfungaia} for the details of the Gaia selection function in order to produce combined selection functions. The reverse dependency shows that studying combined selection function will also lead to constraints on the Gaia selection function. WP\ref{wp:scienceappl} provides the science applications which depend on WPs~\ref{wp:selfungaia} and \ref{wp:selfuncombine} for the selection functions to apply and on WP\ref{wp:selfunimplementation} for the tools to do so. In return WP\ref{wp:scienceappl} provides crucial validation of the results from WPs~\ref{wp:selfungaia} and \ref{wp:selfuncombine} and extensive testing of the implementation provided by WP\ref{wp:selfunimplementation}. WP\ref{wp:selfundefinition} provides the framework for the other WPs as indicated by the grey rectangle.

\begin{figure}
    \centering
    \includegraphics[width=\linewidth]{dependencies.pdf}
    \caption{The dependencies between the work packages in \acro.}
    \label{fig:dependencies}
\end{figure}

\section{Management structure, milestones and procedures}
\label{sec:management}

\subsection{Management}
\label{sec:mgtdetails}

The {\acro} consortium being relatively small the management structure can be kept simple in order to make the coordination of the efforts efficient. The management consists of three levels reflecting the structuring of the {\acro} work packages.
\begin{description}
    \item[Administrative management] This first level concerns the administrative management of {\acro}, including financial management and reporting to the European Commission. It is carried out by the {\acro} coordinator (A.~Brown) with the support of the Leiden Observatory institute management team.
    \item[Overall project management and coordination] This second level concerns the global scientific and technical control and coordination of the project, ensuring that the tasks are properly carried out and remain on schedule, and that the objectives of {\acro} are fulfilled. It is carried out by the \textbf{{\acro} executive board}, formed by the managers of the work packages: A.~Brown, R.~Drimmel, H.-W.~Rix, and V.~Belokurov. In order to keep communications within the consortium as efficient as possible the main contacts at NYU and MONA (D.~Hogg and A.~Casey) will have a standing invitation to attend the executive board meetings (see below).
    \item[Work package management] The third level of management concerns the main work packages. Each WP manager has the responsibility of supervising the execution of the core tasks assigned to their host institution, the work of the partners carrying out specific parts of the tasks attached to the work package, and report to the executive board accordingly.
    
    We do not further specify the management procedures for each work package beyond the items given in \secref{sec:procedures}, since the work in each of the work packages WP\ref{wp:selfundefinition} to WP\ref{wp:scienceappl} will be integrated in already existing and well established teams in each host institution.
\end{description}

This management structure together with the procedures described below is sufficient for a relatively small team spread over only six institutes working together closely in a scientific environment. Much of the coordination and supervision will take place through day-to-day contacts at the institutes involved and between institutes via e-mail and ad-hoc telecons. Similar structures were applied successfully to other EU-funded projects (FP7 GENIUS Collaborative project, FP7 GREAT Marie-Curie Initial Training Network) in which the coordinator was an active participant.

\subsection{Innovation management}
\label{sec:innovationmgmt}

The aim of {\acro} is to develop and deliver a new product (the Gaia selection function and tools to apply it) to the market of astronomy researchers. The wish for a selection function for the Gaia survey has been expressed on numerous occasions, meaning that this product will readily find a large user base. In order to uncover opportunities the astronomical community will be engaged during the lifetime of {\acro} through the availability of a prototype implementation of the Gaia selection function. The users thereof will be asked for feedback in the form of missing features or desired improvements. Community feedback wll also be obtained through presentation of {\acro} results at relevant scientific conferences.

\subsection{Procedures and reporting}
\label{sec:procedures}

We list here the procedures that will apply to the management of {\acro}:
\begin{itemize}
    \item The {\acro} coordinator will report to the European Commission following the rules applicable for research and innovation projects within the Horizon 2020 programme.
    \item {\acro} will hold at least three plenary meetings open to participation of all its members: the kick-off, mid-term, and closing meeting. The participation of the partner institute coordinators (or a representative) will be mandatory. Other plenary meetings may be held if needed.
    \item Specialized meetings or workshops will be organized on an `as-needed' basis by the WP managers.
    \item The executive board will hold teleconferences at least once a month to track the status of the tasks. The minutes of these teleconferences will be made available to the members of the consortium.
    \item The executive board shall ensure the coordination of the {\acro} work with the wider activities on the Gaia mission and its data releases, in particular ensuring the coordination with ESA and DPAC (see \secref{sec:dissemination-exploitation}). This can be done efficiently through membership of several {\acro} team members in DPAC.
\end{itemize}

\subsection{Consortium agreement}
\label{sec:cons_agreement}

The internal workings of {\acro} and the roles and responsibilities of the partners will be formalized through a consortium agreement will be drawn up and agreed at the start of the programme. It will define the decision taking procedures in {\acro} and the mechanisms for conflict resolution, which will be channelled through the Executive Board, as well as the management of intellectual property rights as described in \secref{sec:openipr}.

\subsection{Recruitment strategy}
\label{sec:recruit}

The recruitment policy for {\acro} will conform to the principles of the European Charter for Researchers and the Code of Conduct for their recruitment. It will take place in a globally coordinated way during the six-month setup phase described in \secref{sec:wptiming}, placing an emphasis on individual excellence and capacity for team working, while taking care to ensure equal opportunity and gender balance.

%%%%%%%%%%%%
% MILESTONES
%%%%%%%%%%%%

\milestone[1]{Kick-off meeting}{Organized by {\acro} executive board}{WP\ref{wp:management}}
\milestone[4]{Hiring of researchers}{Partners notify {\acro} executive board}{WP\ref{wp:management}}
\milestone[12]{Document on mathematical formulation of selection function submitted to peer-reviewed journal}{Document approved by {\acro} executive board}{WP\ref{wp:selfundefinition}}
\milestone[18]{Document describing the preliminary Gaia selection function.}{Document approved by {\acro} executive board}{WP\ref{wp:selfungaia}}
\milestone[21]{Mid-term meeting and review}{Evaluation report by {\acro} executive board}{WP\ref{wp:management}}
\milestone[24]{Public prototype of selection function implementation}{Verified by {\acro} executive board}{WP\ref{wp:selfunimplementation}}
\milestone[36]{Papers and documentation describing the Gaia and combined selection functions ready to submit to peer-reviewed journals}{Verified by {\acro} executive board}{WP\ref{wp:selfungaia}, \ref{wp:selfunimplementation}, \ref{wp:selfuncombine}}
\milestone[41]{Closing meeting and community workshop}{Organized by {\acro} executive board}{WP\ref{wp:management}}
\milestone[42]{Publicly available selection function implementation, tools and data}{Verified by {\acro} executive board}{WP\ref{wp:management}, \ref{wp:selfunimplementation}}

\makemilestoneslist

%%%%%%%
% RISKS
%%%%%%%
\criticalrisk{Late recruitment of researchers. Medium}{WP\ref{wp:selfungaia}--\ref{wp:scienceappl}}{Six month preparatory phase added in schedule to accommodate early advertizing of open positions. Use professional network of {\acro} participants. Executive board will track recruitment.}
\criticalrisk{Recruited researchers leaving early. Medium}{WP\ref{wp:selfungaia}--\ref{wp:scienceappl}}{Keep staff motivated, including through supporting career development. Coordinator and executive board will monitor personnel happiness.}
\criticalrisk{Unexpected complexity of tasks. High}{WP\ref{wp:selfungaia}--\ref{wp:scienceappl}}{Prioritize the {\acro} efforts and make timely decisions on dropping tasks deemed to complex.}
\criticalrisk{Lack of information from DPAC. Low}{WP\ref{wp:selfungaia}--\ref{wp:selfuncombine}}{{\acro} team members will be embedded in DPAC. Coordinator can identify alternative routes within DPAC to obtaining information.}
\criticalrisk{Coordination problems with ESA and DPAC leading to the selection function data products not being available through ESA archives. Low}{WP\ref{wp:selfunimplementation}}{Plan for alternative distribution channels. Coordinator to maintain close connections to ESA and DPAC.}
\criticalrisk{Delay of Gaia data releases. Can lead to difficulties in coordinating with DPAC/ESA (due to their prioritizing working on the releases). Medium}{WP\ref{wp:selfungaia}--\ref{wp:scienceappl}}{Coordinator to monitor the situation and agree mitigation measures with ESA/DPAC. We stress that even in the extremely unlikely case that no further Gaia data releases were to happen, the {\acro} effort will still have a large positive impact on the exploitation of the existing Gaia DR2 data.}

\makerisklist

\section{Consortium as a whole}
\label{sec:consortium}

The idea for this proposal grew out of discussions at several Gaia Sprints\footnote{\url{http://gaia.lol}} in which all of the {\acro} members participated, Hogg, Casey, Price-Whelan, and Fouesneau being members of the organizing team of the Sprints. In particular at the Gaia Sprint that took place in Santa Barbara (USA) in March 2019, two sessions were dedicated solely to discussing the Gaia selection function. It was concluded that only a significant and dedicated effort, specifically funded, could realistically lead to the construction of the Gaia selection function and the corresponding tools and data needed to use it.

The consortium represents a combination of Gaia/DPAC experts (Brown, Drimmel, Fouesneau), experts that have given much thought to the principles of selections functions and worked on aspects of the Gaia selection function (Hogg, Rix, Drimmel, Price-Whelan, Casey, Fouesneau). All the participants have undertaken scientific analyses of the Gaia data that clearly highlight the need for a selection function. \memo{Look up relevant publications.}
\begin{itemize}
    \item The Gaia expertise in the consortium is contributed by long-standing DPAC members who are deeply involved in the data processing for the Gaia mission. They provide expert insight into the detailed ingredients of the Gaia selection function and, more importantly, have an extensive network of contacts with experts in DPAC who can provide more detailed information where needed. The presence of the Gaia experts is crucial to the guidance of the researchers to be funded through {\acro}. We note that the other participants have extensive experience as users of Gaia data and thus bring an important complementary perspective to {\acro}.
    \item The expertise on the mathematical formulation of the selection function and on selection functions in general is essential to the success of {\acro}. The expertise stems from the actual construction of selection functions and applying these to analyses of a number of large surveys \memo{(Example citations)}.
    \item The requirements on a selection function and how it is provided in its practical form must be guided by the experience gained through advanced scientific analyses of data such as from the Gaia mission. The experience from successes and failures in past analyses will be very important in setting the priorities for the research \& development in {\acro}.
\end{itemize}

The partners from Australia and the USA contribute deep knowledge of selection function issues based on their past scientific work. Their perspective as experienced users of the Gaia data will be essential to ensure that the developments in {\acro} stay focused on satisfying the needs of the astronomical community in general, and not just the needs of DPAC experts.

The participants in {\acro} have a long track record of effective collaboration as can be seen from the many joint publications. In addition good connections exist to experts who have already taken concrete steps to providing a Gaia selection function implementation, such as Boubert \& Everall\footnote{\url{https://github.com/DouglasBoubert/selectionfunctions}} (Oxford, Cambridge) and Rybizki\footnote{\url{https://github.com/jan-rybizki/gdr2_completeness}} (Heidelberg). Through the organization of Gaia Sprints and other workshops\footnote{For example the Gaia DR2 Exploration Lab \url{https://www.cosmos.esa.int/web/gaia-dr2-exploration}.} the participants have a proven track record of engaging the astronomical community, an important aspect of making sure that the tool and data delivered by {\acro} fulfil actual needs.

\begin{itemize}
    \item Motivate why this consortium.
    \item Relations to ESA and DPAC and how these strengthen the consortium.
    \item Explain why US/Australia are involved.
    \item Refer to Gaia Sprints and the Santa Barbara discussion on selection functions
    \item Links to MW-Gaia COST Action and MWGaiaITN
\end{itemize}

\section{Resources to be committed}
\label{sec:resources}

\makesummaryofefforttable

\costsTravel{ULEI}{2500}{Explain}
\costsEquipment{ULEI}{3000}{Not needed probably}
\costsOther{ULEI}{60000}{TBD}

\makecoststable

%%% Local Variables:
%%% mode: latex
%%% TeX-master: "proposal-main"
%%% End:
