\chapter{Implementation}
\label{cha:implementation}

\section{Work plan --- Work packages, deliverables}
\label{sec:work-plan}

\subsection{Overall structure}
\label{sec:wpstructure}

The overall approach to the work plan is to have the research and development of the selection function progress in parallel with the implementation thereof. We made this choice because of the complexities of the selection function and its realization in the form of tools and auxiliary data, and because of the relatively short duration of the project. The work flow we foresee for {\acro} is shown in \figref{fig:workflow}. It is an iterative workflow in which the findings in each of work packages \ref{wp:selfungaia}--\ref{wp:scienceappl} will inform the others. For example the research on the selection function will lead to requirements on the implementation, while at the same time the conversion of research ideas into practical tools will lead to the questioning and revision of those same ideas. At several points during the project the results will be combined into prototype implementations of the selection function tools. These will be tested both internally to the project and through the public availability of one prototype implementation. The results from the tests will feed into the next round of research, development, and implementation. The scientific nature of much of the work conducted in {\acro} will very much benefit from an iterative approach. Close coordination between the work packages will of course be essential and this is foreseen in the management structure.

\begin{figure}
    \centering
    \includegraphics[width=\linewidth]{workflow.pdf}
    \caption{Iterative workflow for CoG.}
    \label{fig:workflow}
\end{figure}

\begin{figure}[ht!]
    \centering
    \includegraphics[width=\linewidth]{cog-gantt.pdf}
    \caption{Gantt chart for the {\acro} project.}
    \label{fig:gantt}
\end{figure}

\subsection{Timing of the work packages}
\label{sec:wptiming}

The overall work plan is structured as follows:
\begin{itemize}
    \item The first 6 months will be used to recruit the researchers to be funded from this proposal and to carry out the tasks in work package \ref{wp:selfundefinition}. Having a mathematical framework for the selection function in place early on in the project is important for guiding the research and the implementation phases.
    \item During the next three years of the project the efforts in work packages \ref{wp:selfungaia} to \ref{wp:scienceappl} will happen mostly in parallel for the reasons outlined above. The iterative approach outline in \secref{sec:wpstructure} will be used in this phase.
 \end{itemize}

\makewplist

%\subsection{Work package description}
%\label{sec:wps}

\tablecaption{Description of work packages}
\begin{supertabular}{p{\textwidth}}
    \omit \tabularnewline
\end{supertabular}
%%%%%%%%%%%%%%%%%%%%%%%%%%%%%%
%  Work Package Description  %
%%%%%%%%%%%%%%%%%%%%%%%%%%%%%%

\begin{workpackage}{Management}
  \label{wp:management} %change and use appropriate description

  %%%%%%%%%%%%%%%%%% TOP TABLE %%%%%%%%%%%%%%%%%%%%%%%%%%%%%
  % Data for the top table
  \wpstart{1} %Starting Month
  \wpend{36} %End Month
  \wptype{Activity type} %RTD, DEM, MGT, or OTHER
  \wplead{ULEI}

  % Person Months per participant (required, max 7, * for leader)  
  % syntax: \personmonths{Participant number}{value}    (not wp leader)
  %     or  \personmonths{Participant short name}{value} (not wp leader)
  %         \personmonths*{Participant number}{value}    (wp leader)
  % for example:
  \personmonths*{ULEI}{12}
  \personmonths{MPIA}{3}
  \personmonths{INAF}{2}
  % etc.

  \makewptable % Work package summary table

  % Work Package Objectives
  \begin{wpobjectives}
    This work package has the following objectives:
    \begin{enumerate}
      \item To develop ....
      \item To apply this ....
      \item etc.
    \end{enumerate}
  \end{wpobjectives}

  % Work Package Description
  \begin{wpdescription}
    % Divide work package into multiple tasks.
    % Use \wptask command
    % syntax: \wptask{leader}{contributors}{start-m}{end-m}{title}{description}   

    Description of work carried out in WP, broken down into tasks, and
    with role of partners list. Use the \texttt{\textbackslash wptask} command.

    \wptask{ULEI}{ULEI}{1}{12}{Test}{
      \label{task:wp1test}
      Here we will test the WP Task code. 
    }
    \wptask{ULEI}{All other}{6}{9}{Integrate}{
      \label{task:wp1integrate}
      In this task UZH will integrate the work done in ~\ref{task:wp1test}.
    }    
    \wptask{ULEI}{All other}{9}{12}{Apply}{
      Here all the WP participants will apply the results to...
    }

    \paragraph{Role of partners}
    \begin{description}
      \item[Participant short name] will lead Task~\ref{task:wp1integrate}.
      \item[UoC] will..
    \end{description}
  \end{wpdescription}

  % Work Package Deliverable
  \begin{wpdeliverables}
    % Data for the deliverables and milestones  tables
    % syntax: \deliverable[delivery date]{nature}{dissemination
    % level}{description} 
    %
    % nature: R = Report, DEM = Demonstrator, DEC = Websites, media, etc, OTHER = Other
    % dissemination level: PU = Public, CO = Confidential, CI = CLassified.
    % 
    % \wpdeliverable[date]{R}{PU}{A report on \ldots}

    \wpdeliverable[6]{ULEI}{R}{PU}{Data management plan}\label{dev:wp1datamanagement}

    \wpdeliverable[12]{ULEI}{R}{PU}{Report on Feasibility study for the model
    implementation.}\label{dev:wp1implementation}

    \wpdeliverable[24]{ULEI}{R}{PU}{Prototype of model
    implementation.}\label{dev:wp1prototype}

  \end{wpdeliverables}

\end{workpackage}


%%% Local Variables:
%%% mode: latex
%%% TeX-master: "proposal-main"
%%% End:

\begin{workpackage}{Definition of the selection function}
  \label{wp:selfundefinition}
  \wpstart{1} %Starting Month
  \wpend{12} %End Month
  \wptype{RTD} %RTD, DEM, MGT, or OTHER
  \wplead{MPG}
  \personmonths{ULEI}{1}
  \personmonths*{MPG}{2}
  \personmonths{INAF}{1}
  \personmonths{UCAM}{1}
  \personmonths{NYU}{1}
  \personmonths{MONA}{1}
  
  \makewptable % Work package summary table

  % Work Package Objectives
  \begin{wpobjectives}
    This objective of this work package is to research and implement a precise mathematical formulation of the concept of a survey selection function. The results will be written up as a scientific paper (to be published in the peer-reviewed literature) that will guide the rest of the work to be done within {\acro}.  The definition of the selection function will account for the following aspects:
    \begin{itemize}
        \item 
    \end{itemize}
  \end{wpobjectives}

  % Work Package Description
  \begin{wpdescription}
    % Divide work package into multiple tasks.
    % Use \wptask command
    % syntax: \wptask{leader}{contributors}{start-m}{end-m}{title}{description}   

    Description of work carried out in WP, broken down into tasks, and
    with role of partners list. Use the \texttt{\textbackslash wptask} command.

    \wptask{UCAM}{UCAM}{1}{12}{Test}{
      \label{task:wp2test}
      Here we will test the WP Task code. 
    }
    \wptask{ULEI}{All other}{6}{9}{Integrate}{
      \label{task:wp2integrate}
      In this task UZH will integrate the work done in ~\ref{task:wp2test}.
    }    
    \wptask{ULEI}{All other}{9}{12}{Apply}{
      Here all the WP participants will apply the results to...
    }

    \paragraph{Role of partners}
    \begin{description}
      \item[Participant short name] will lead Task~\ref{task:wp2integrate}.
      \item[UoC] will..
    \end{description}
  \end{wpdescription}

  % Work Package Deliverable
  \begin{wpdeliverables}
    % \wpdeliverable[date]{R}{PU}{A report on \ldots}

    \wpdeliverable[36]{UCAM}{R}{PU}{Report on the definition of the model
    specifications.}\label{dev:wp2specs}

    \wpdeliverable[12]{ULEI}{R}{PU}{Report on Feasibility study for the model
    implementation.}\label{dev:wp2implementation}

    \wpdeliverable[24]{ULEI}{R}{PU}{Prototype of model
    implementation.}\label{dev:wp2prototype}

  \end{wpdeliverables}

\end{workpackage}


%%% Local Variables:
%%% mode: latex
%%% TeX-master: "proposal-main"
%%% End:

\begin{workpackage}{Research and develop the Gaia selection function}
  \label{wp:selfungaia}
  \wpstart{7} %Starting Month
  \wpend{\duration} %End Month
  \wptype{RTD} %RTD, DEM, MGT, or OTHER
  \wplead{UCAM}
  \personmonths{ULEI}{2}
  \personmonths{MPIA}{6}
  \personmonths{INAF}{10}
  \personmonths*{UCAM}{23}
  \personmonths{NYU}{0}
  \personmonths{MONA}{0}
  
  \makewptable % Work package summary table

  % Work Package Objectives
  \begin{wpobjectives}
    The objective of this work package is to research and develop in detail the description and modelling of the Gaia survey selection function.
    \begin{enumerate}
      \item Develop the overall survey selection function and using that as a starting point create more specialized selection functions, focusing for example the Gaia astrometric, photometric, and spectroscopic surveys and combinations thereof. In addition a selection function for specific subsets of the Gaia survey will be developed, such as binary stars and exoplanets, variable stars, solar system objects, extragalactic sources. It will be essential here to agree to the scope of the work early on, where the development of further specialized Gaia selection function is left to the community who can make use of the tools developed in this work package. 
      \item Investigate to what extent detailed information is needed on the Gaia pointing history, its on-board detection algorithm, and the data losses introduced along the various steps from on-board measurement to the final data products. Can the selection function be reverse engineered from publicly available information?
      \item Interface with DPAC to gain a deeper understanding of the many ingredients of the Gaia selection function, in particular of aspects where public information is not (yet) available. \memo{This should be formulated more clearly.}
    \end{enumerate}
  \end{wpobjectives}

  % Work Package Description
  \begin{wpdescription}
    % Divide work package into multiple tasks.
    % Use \wptask command
    % syntax: \wptask{leader}{contributors}{start-m}{end-m}{title}{description}   

    \wptask{UCAM}{UCAM}{7}{\duration}{Scientific coordination}{
      \label{task:wp3coordination}
       Scientific coordination of the research and development of the selection function and coordination of writing the paper corresponding to deliverable \ref{dev:wp3GSFfinal}.
    }
    \wptask{UCAM}{All other}{7}{\duration}{R\&D top level selection functions}{
      \label{task:wp3toplevelGSF}
      Research and development of the top-level Gaia selection function which will describe the probability that a source enters the Gaia astrometric, photometric or radial velocity surveys, and combinations thereof.
    }    
    \wptask{MPIA}{All other}{7}{\duration}{R\&D spectroscopic selection functions}{
      \label{task:wp3spectroscopicGSF}
      Research and development of the Gaia spectroscopic selection function. This refers to the probability that a BP/RP and/or and RVS spectrum was measured for a source in the Gaia survey.
    }
    \wptask{INAF}{All other}{7}{\duration}{R\&D specialized selection functions}{
      \label{task:wp3specializedGSF}
      Research and develop the methods needed to create selection functions for specific subsets of the Gaia survey, such as binary stars or extragalactic objects. Apply this to example cases.
    }

    \paragraph{Role of partners}
    \begin{description}
      \item[UCAM] will lead Tasks~\ref{task:wp3coordination} and \ref{task:wp3toplevelGSF}.
      \item[MPIA] will lead Task~\ref{task:wp3toplevelGSF}.
      \item[INAF] will lead Task~\ref{task:wp3specializedGSF}
      \item[All partners] will contribute to the research and development of the Gaia selection functions. 
    \end{description}
  \end{wpdescription}

  % Work Package Deliverable
  \begin{wpdeliverables}
    % \wpdeliverable[date]{R}{PU}{A report on \ldots}
    \wpdeliverable[18]{UCAM}{R}{PU}{Documentation of preliminary top level Gaia selection functions.}\label{dev:wp3version1selfun}
    \wpdeliverable[36]{UCAM}{R}{PU}{Documentation of the Gaia selections functions (also to be submitted to as papers to a peer-reviewed journal)}\label{dev:wp3GSFfinal}
    \memo{Add further deliverables? Such as intermediate reports for each of the tasks above?}
  \end{wpdeliverables}

\end{workpackage}


%%% Local Variables:
%%% mode: latex
%%% TeX-master: "proposal-main"
%%% End:

\begin{workpackage}{Practical implementation and dissemination of the Gaia selection function}
  \label{wp:selfunimplementation}
  \wpstart{7} %Starting Month
  \wpend{\duration} %End Month
  \wptype{RTD} %RTD, DEM, MGT, or OTHER
  \wplead{ULEI}
  \personmonths*{ULEI}{23}
  \personmonths{MPG}{6}
  \personmonths{INAF}{2}
  \personmonths{UCAM}{2}
  \personmonths{NYU}{0}
  \personmonths{MONA}{0}
 
  \makewptable % Work package summary table

  % Work Package Objectives
  \begin{wpobjectives}
    This work package provides the practical implementation of the Gaia selection functions in terms of data and associated source code. Dissemination of these results through the ESA Gaia archive and code hosting web-sites is also part of this work package. The detailed objectives are:
    \begin{enumerate}
      \item Implement the selection function as defined and developed in detail in work packages \ref{wp:selfundefinition} and \ref{wp:selfungaia} in the form of open source code and associated numerical data (in the form of tables or any other convenient format).
      \item Implement a tool that allows for layering user defined selections on the Gaia archive data on top of the survey selection functions.
      \item Provide implementations of a number of combined selection functions for intersections of Gaia and selected other surveys.
      \item Identify code hosting options and make the Gaia selection function source code available publicly.
      \item Agree with ESA/DPAC on the hosting of the numerical data associated with the Gaia selection function in the ESA archive ecosystem.
    \end{enumerate}
  \end{wpobjectives}

  % Work Package Description
  \begin{wpdescription}
    % Divide work package into multiple tasks.
    % Use \wptask command
    % syntax: \wptask{leader}{contributors}{start-m}{end-m}{title}{description}
    \wptask{ULEI}{ULEI}{1}{\duration}{Scientific coordination}{
      \label{task:wp4coordination}
      Scientific coordination of the implementation and dissemination of the Gaia survey selection function. This includes setting up the code hosting and interfacing with ESA/DPAC on hosting the numerical data for the selection function in the Gaia archive.
    }
    \wptask{ULEI}{All other}{7}{\duration}{Implementation}{
      \label{task:wp4implement}
      Implement the Gaia selection functions in terms of open source software tools and associated numerical data. Test the implementation through applications to science cases. Port the implementation to the relevant code hosting site and transfer the numerical data to the ESA Gaia archive.
    }    
    \wptask{MPG}{All other}{7}{\duration}{Implementation of tools to include user selections}{
      \label{task:wp4layers}
      Develop and implement tools to chain together the Gaia survey selection function with additional user imposed selection on the Gaia archive data. The tools should produce an overall effective selection function for the user-selected sample.
    }
    \wptask{INAF}{All other}{7}{\duration}{Implementation of tools for combined selection functions}{
      \label{task:wp4combine}
      Implement the selection functions for combinations of Gaia and other surveys based on the methodology developed in work package \ref{wp:selfuncombine}.
    }

    \paragraph{Role of partners}
    \begin{description}
      \item[ULEI] will lead Task~\ref{task:wp4coordination}.
      \item[MPG] will lead Task~\ref{task:wp4layers}.
      \item[INAF] will lead Task~\ref{task:wp4combine}.
      \item[All partners] will contribute to the implementation and testing of the tools developed in this work package. In particular the science applications pursued by the partners will constitute a strong test of the Gaia selection function implementation.
    \end{description}
  \end{wpdescription}

  % Work Package Deliverable
  \begin{wpdeliverables}
    % \wpdeliverable[date]{R}{PU}{A report on \ldots}
    \wpdeliverable[24]{ULEI}{DEM}{PU}{Prototype of open source and open data implementation of Gaia selection function.}\label{dev:wp4prototype}
    \wpdeliverable[\duration]{ULEI}{DEC}{PU}{Open source and open data implementation of Gaia selection function.}
    \memo{Probably need a few more deliverables as concrete checks on progress.}
  \end{wpdeliverables}

\end{workpackage}


%%% Local Variables:
%%% mode: latex
%%% TeX-master: "proposal-main"
%%% End:

\begin{workpackage}{Selection function for combinations of Gaia and other surveys}
  \label{wp:selfuncombine}
  \wpstart{7} %Starting Month
  \wpend{\duration} %End Month
  \wptype{RTD} %RTD, DEM, MGT, or OTHER
  \wplead{MPG}
  \personmonths{ULEI}{2}
  \personmonths*{MPG}{15}
  \personmonths{INAF}{12}
  \personmonths{UCAM}{2}
  \personmonths{NYU}{0}
  \personmonths{MONA}{0}

  \makewptable % Work package summary table

  % Work Package Objectives
  \begin{wpobjectives}
    The objective of this work package is to research and develop methods to derive selection functions for the combination of Gaia and other large sky surveys. A concrete implementation for a few cases will be implemented in work package \ref{wp:selfunimplementation}. The detailed objectives are
    \begin{enumerate}
      \item Research and develop a generic method for constructing selection functions for combinations of surveys.
      \item Apply this method to a few concrete cases of the combination of Gaia with another survey.
    \end{enumerate}
  \end{wpobjectives}

  % Work Package Description
  \begin{wpdescription}
    % Divide work package into multiple tasks.
    % Use \wptask command
    % syntax: \wptask{leader}{contributors}{start-m}{end-m}{title}{description}   
    \wptask{MPG}{All other}{7}{\duration}{Scientific coordination.}{
      \label{task:wp5coordination}
      Scientific coordination of the research and development of the combined selection function and coordination of writing the paper corresponding to deliverable \ref{dev:wp5finalreport}.
    }
    \wptask{MPG}{All other}{7}{\duration}{Generic combination method.}{
      \label{task:wp5method}
      Research and develop a generic method for constructing selection functions for combinations of surveys.
    }
    \wptask{MPG}{All other}{7}{\duration}{Example combined selection functions.}{
      \label{task:wp5examples}
      For a few selected examples construct the selection function for the combination of Gaia and another survey, using the method developed in task~\ref{task:wp5method}.
    }    

    \paragraph{Role of partners}
    \begin{description}
      \item[MPG] will lead Tasks~\ref{task:wp5coordination}, \ref{task:wp5method}, and \ref{task:wp5examples}.
      \item[All other partners] will contribute to tasks \ref{task:wp5method} and \ref{task:wp5examples}.
    \end{description}
  \end{wpdescription}

  % Work Package Deliverable
  \begin{wpdeliverables}
    % \wpdeliverable[date]{R}{PU}{A report on \ldots}
    \wpdeliverable[\duration]{MPG}{R}{PU}{Report documenting the methods to construct combined selection functions (also to be submitted to peer-reviewed journal).}\label{dev:wp5finalreport}
  \end{wpdeliverables}

\end{workpackage}


%%% Local Variables:
%%% mode: latex
%%% TeX-master: "proposal-main"
%%% End:

\input{wp-scienceappl}

\makedeliverablelist

\subsection{Dependencies}
\label{sec:dependencies}


\section{Management structure, milestones and procedures}
\label{sec:management}

\subsection{Management}
\label{sec:mgtdetails}

\subsection{Procedures and reporting}
\label{sec:procedures}

\subsection{Consortium agreement}
\label{sec:cons_agreement}

\subsection{Recruitment strategy}
\label{sec:recruit}

\milestone[1]{Kick-off meeting}{Organized by {\acro} board}{WP\ref{wp:management}}
\milestone[12]{Document on mathematical formulation of selection function submitted to peer-reviewed journal}{Document approved by {\acro} board}{WP\ref{wp:selfundefinition}}

\makemilestoneslist

\criticalrisk{Risk to this project.}{WP\,\ref{wp:management}}{Mitigation measure}

\makerisklist

\section{Consortium as a whole}
\label{sec:consortium}

\begin{itemize}
    \item Motivate why this consortium.
    \item Relations to ESA and DPAC and how these strengthen the consortium.
    \item Explain why US/Australia are involved.
    \item Refer to Gaia Sprints and the Santa Barbara discussion on selection functions
    \item Links to MW-Gaia COST Action and MWGaiaITN
\end{itemize}

\section{Resources to be committed}
\label{sec:resources}

\makesummaryofefforttable

\costsTravel{ULEI}{2500}{Explain}
\costsEquipment{ULEI}{3000}{Not needed probably}
\costsOther{ULEI}{60000}{TBD}

\makecoststable

%%% Local Variables:
%%% mode: latex
%%% TeX-master: "proposal-main"
%%% End:
