\chapter{Impact}
\label{cha:impact}


\section{Expected impacts} 
\label{sec:expected-impact}
\instructions{
\begin{itemize}
    \item Describe how your project will contribute to:
    \begin{itemize}
        \item each of the expected impacts mentioned in the work programme, under the relevant topic;
        \item any substantial impacts not mentioned in the work programme, that would enhance innovation capacity; create new market opportunities, strengthen competitiveness and growth of companies, address issues related to climate change or the environment, or  bring other important benefits for society
    \end{itemize}
    \item Describe any barriers/obstacles, and any framework conditions (such as regulation, standards, public acceptance, workforce considerations, financing of follow-up steps, cooperation of other links in the value chain), that may determine whether and to what extent the expected impacts will be achieved. (This should not include any risk factors concerning implementation, as covered in section 3.2.)
\end{itemize}
}

\begin{itemize}
    \item Expertise gained in this project can be applied to other/future large surveys and space missions.
    \item Data and tools for incorporating selection function in your science.
\end{itemize}

\section{Measures to maximize impact} 
\label{sec:maximize-impact}

\begin{itemize}
    \item Publication of results, including science applications.
    \item Data availability through ESA archives, code through code hosting sites.
    \item Organize workshops to demo the selection function and help astronomers with their concrete problems.
\end{itemize}

\subsection{a) Dissemination and exploitation of results}
\label{sec:dissemination-exploitation}
\instructions{
\begin{itemize}
    \item Provide a draft `plan for the dissemination and exploitation of the project's results'. Please note that such a draft plan is an admissibility condition, unless the work programme topic explicitly states that such a plan is not required.\\
    Show how the proposed measures will help to achieve the expected impact of the project.\\
    The plan, should be proportionate to the scale of the project, and should contain measures to be implemented both during and after the end of the project. For innovation actions, in particular, please describe a credible path to deliver these innovations to the market.
    \item Include a business plan where relevant.
    \item As relevant, include information on how the participants will manage the research data generated and/or collected during the project, in particular addressing the following issues:
    \begin{itemize}
        \item What types of data will the project generate/collect? o What standards will be used? 
        \item How will this data be exploited and/or shared/made accessible for verification and re-use? If data cannot be made available, explain why.
        \item How will this data be curated and preserved? 
        \item How will the costs for data curation and preservation be covered?
    \end{itemize}
    \item Outline the strategy for knowledge management and protection. Include measures to provide open access (free on-line access, as the `green' or `gold' model) to peer-reviewed scientific publications which might result from the project.
\end{itemize}
}

\begin{itemize}
    \item Open source policy for the code/tools
        \begin{itemize}
            \item Host on e.g. Zenodo, Github, \ldots
            \item \memo{note: hosting at ESA implies one has to be extremely careful and consistent with the
                licenses, which for academic code will quickly become a major hassle}
        \end{itemize}
    \item Open access for all data and code produced in this project
        \begin{itemize}
            \item Host these in Gaia archive if possible
        \end{itemize}
\end{itemize}

\subsection{b) Communication activities}
\label{sec:communication-activities}
\instructions{
\begin{itemize}
    \item Describe the proposed communication measures for promoting the project and its findings during the period of the grant. Measures should be proportionate to the scale of the project, with clear objectives.  They should be tailored to the needs of different target audiences, including groups beyond the project's own community.
\end{itemize}
}

\begin{itemize}
    \item Academic outreach through publications, presentations at conferences and the above mentioned workshop(s).
    \item Public outreach through results from the science applications in this project.
\end{itemize}
