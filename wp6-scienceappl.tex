\begin{workpackage}{Science applications}
  \label{wp:scienceappl}
  \wpstart{1} %Starting Month
  \wpend{\duration} %End Month
  \wptype{RTD} %RTD, DEM, MGT, or OTHER
  \wplead{INAF}
  \personmonths{ULEI}{13}
  \personmonths{MPIA}{13}
  \personmonths*{INAF}{15}
  \personmonths{UCAM}{13}
  \personmonths{NYU}{1}
  \personmonths{MONA}{1}

  \makewptable % Work package summary table

  % Work Package Objectives
  \begin{wpobjectives}
    The objective of this work package is to apply the Gaia selection function to four science cases. This serves to test the Gaia selection function implementation, to provide the community with worked examples of how to use the Gaia selection function, and to motivate the researchers involved in {\acro} and ensure that they have a scientific publication to enhance their prospects for future jobs in academia. The four science cases are:
    \begin{enumerate}
      \item \textbf{Binary stars in Gaia}{
        Binary stars have a complex dependency on the Gaia selection function that depends on the source observables (e.g. colour and apparent magnitude, see Fig.\ref{fig:binaries}) and the orbital properties (e.g. period, inclination, mass ratio). In this application we would apply the selection functions for Gaia astrometry and spectroscopy to understand what Gaia would, or would not, observe for a fictitious population of binary stars. Given the selection functions we would research how the Gaia observables can be used to detect and partially characterise binary star systems, and more importantly to understand which systems would not have been detected or characterised. 

        By applying the selection functions to example populations of binary stars (e.g. using COMPAS\footnote{https://compas.science/}, which is used and developed by a large group at MONA) we will perform population inference to make appropriate comparisons between models and the Gaia catalog. With unbiased characterisation of binary star populations, and comparisons between observations and models where the selection functions are appropriately applied, we expect to provide unbiased answers to many long-standing questions in stellar multiplicity research, including: how does stellar multiplicity change with metallicity and the surrounding environment? What is the distribution of initial mass ratios of binary systems? How do tidal effects change the evolution of stars? 

        With an unbiased characterisation of the population of binary star systems, and comparisons with models where the Gaia selection functions have been applied, we expect to address questions like how stellar multiplicity changes with metallicity, and understanding the initial mass ratio distribution of stars, the timescales of tidal circularisation, to name but a few.}
      \item \textbf{The Oort Limit, and Dark Matter near the Sun}{
      The positions and motions of stars near (<500~pc) the Sun, can constrain the gravitational potential at the Solar radius, and -- when compared to a direct census of the stellar mass -- can inform us about dark matter near the Sun. This is an analysis that has been carried out with ground-based data since the pioneering work of Oort \citep[e.g.][]{Read2014}. The robustness of results has been hampered by the fact that the local dark matter density is the difference bnetween the total one (measured by dynamics) and the stellar one, obtained by stars count (or CMD analysis); Gaia can do far better on both aspects, leading to a much better "difference" of the two terms. 
      
      The dynamical mass (to be measured to $<10$\%) requires exceptionally good knowledge of the selection function, as star-tracer densities and their derivatives enter in the analysis (see Eq.\ref{eqn:Jeans}). The census of stellar mass can happen via placing stars onto the CMD (using parallaxes and precise colors), inferring each stars' mass and adding them up. But in any magnitude limited sample, the mix of luminous to faint stars will change as a function of distance: both dust extinction and the selection function will need to be modelled precisely. 
      
      This foreseen paper will aim  to provide a benchmark on how to incorporate the Gaia selection function into a dynamical analysis (and tell us how much dark matter there is near the Sun.}
      
     
      \item \textbf{The Luminosity Function of White Dwarfs}{
      White dwarfs (WD) are fascinating stellar astrophysics laboratories, the presumed progenitors of type Ia Supernovae, and precision clocks for Milky Way evolution; and WD fade, so the volume over which they can be seen in practical survey, depends on their age (and mass).
      Gaia DR2 has provided the community with an unprecedented sample of WD \citep{WD_DR2}, hot stellar sources of low luminosity (or high parallax).
      When CMD (+ stellar model) age-dated, these white dwarfs could be an unparalleled approach to understand the age distribution of the oldest disk stars, their vertical motions, etc..
      
      To put this in practice, the selection function is needed, here also the positional dependence of the parallax uncertainty, as ``good'' parallaxes are indispensable to identify them in the first place. We will set out to use the selection function derived in this proposal to derive a) the luminosity function of white dwarfs in different volumes around the Sun, and the spatial (i.e. vertical) distribution and velocity, as a function of its CMD-age. This can then provide a foundation for studies of Galactic disk evolution, and (!) of white dwarf cooling physics.
      }
      \item Science application D\ldots
    \end{enumerate}
    \memo{These are still open. Suggestions and corresponding text welcome.}
  \end{wpobjectives}

  % Work Package Description
  \begin{wpdescription}
    % Divide work package into multiple tasks.
    % Use \wptask command
    % syntax: \wptask{leader}{contributors}{start-m}{end-m}{title}{description}   
    \wptask{INAF}{INAF}{1}{\duration}{Scientific coordination.}{
      \label{task:wp6coordination}
      Coordination of the efforts on the science applications of the Gaia selection function. This includes monitoring the progress of the scientific papers, organizing the necessary meetings to discuss the paper contents, ensuring that the papers expose the use of the main elements of the tools developed in {\acro}.
      
      \textsf{2 INAF \pems}
    }
    \wptask{ULEI}{ULEI, MONA}{7}{\duration}{Science application A.}{
      \label{task:wp6sciA}
      Carry out the research related to science application A and write a paper for a peer-reviewed journal.
      
      \textsf{13 ULEI + 1 MONA \pems}
    }
    \wptask{MPIA}{MPIA, NYU}{7}{\duration}{Science application B.}{
      \label{task:wp6sciB}
      Carry out the research related to science application B and write a paper for a peer-reviewed journal.
      
      \textsf{13 MPIA + 1 NYU \pems}
    }
    \wptask{INAF}{INAF}{7}{\duration}{Science application C.}{
      \label{task:wp6sciC}
      Carry out the research related to science application C and write a paper for a peer-reviewed journal.
      
      \textsf{13 INAF \pems}
    }
    \wptask{UCAM}{UCAM}{7}{\duration}{Science application D.}{
      \label{task:wp6sciD}
      Carry out the research related to science application D and write a paper for a peer-reviewed journal.
      
      \textsf{13 UCAM \pems}
    }

    \paragraph{Role of partners}
    \begin{description}
      \item[ULEI] will lead Task~\ref{task:wp6sciA}.
      \item[MPIA] will lead Task~\ref{task:wp6sciB}.
      \item[INAF] will lead Task~\ref{task:wp6sciC}.
      \item[UCAM] will lead Task~\ref{task:wp6sciD}.
      \item[All partners] will contribute to each of the tasks \ref{task:wp6sciA}--\ref{task:wp6sciD}.
    \end{description}
    The researchers funded through {\acro} are expected to do the bulk of the scientific analysis for the tasks above, supported by local supervision.
  \end{wpdescription}

  % Work Package Deliverable
  \begin{wpdeliverables}
    % \wpdeliverable[date]{Participant}{R}{PU}{A report on \ldots}
    \memo{Not sure we should put deliverables here. MF: Would this be papers then?}
  \end{wpdeliverables}

\end{workpackage}


%%% Local Variables:
%%% mode: latex
%%% TeX-master: "proposal-main"
%%% End:
