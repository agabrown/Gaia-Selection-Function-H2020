\subsection{Universiteit Leiden}
\label{sec:ulei}

\paragraph{General description}
Leiden Observatory is the astronomical institute of the Faculty of Science of Leiden University. Established in 1633, it is the oldest university observatory in operation today, with a very rich tradition. Leiden Observatory carries out world class research in the formation of structures in the universe and the origin and evolution of galaxies, the detection and characterization of exoplanets, and the formation of stars and planetary systems. The institute consists of about 35 faculty and adjunct faculty, 50 postdoctoral researchers, 50 MSc and 80 PhD students, and 30 support staff. Leiden Observatory offers an excellent educational programme at the Bachelor’s and Master’s levels and a renowned PhD programme. Within the Faculty of Science, the institute closely collaborates with the Leiden Institute of Physics, the Mathematical Institute and the Leiden Institute of Advanced Computer Science.

\paragraph{Key persons}
Dr.~Anthony Brown (male) is a faculty member of Leiden Observatory. He has published over 75 papers in the peer-reviewed literature, and supervised 9 PhD students and several postdocs. He has been deeply involved with the Gaia mission since 1997, having contributed to the optimization of the mission, the preparations for the data processing, and the implementation of several algorithms in the DPAC data processing pipelines. He currently serves as the chair of the Gaia Data Processing and Analysis Consortium and is a member of the Gaia Science Team. The efforts described in this proposal will very much benefit from the extensive expertise of Brown on both the Gaia spacecraft and its instruments as well as the DPAC processing pipelines. The close connections to ESA and DPAC through Brown's role in the latter will facilitate the communications regarding details of the selection function, the integration of the results in the Gaia archive, and the transfer of expertise to DPAC.

\paragraph{Relevant publications}
\begin{itemize}
    \item Gaia Collaboration, \textbf{Brown}, et al., 2018, Gaia Data Release 2. Summary of the contents and survey properties, A\&A 616, A1 
    \item Snellen \& \textbf{Brown}, 2018, The mass of the young planet Beta Pictoris b through the astrometric motion of its host star, Nature Astronomy 2, 883 
    \item Zari, Hashemi, \textbf{Brown}, et al., 2018, 3D mapping of young stars in the solar neighbourhood with Gaia DR2, A\&A 620, A172 
    \item  Luri, \textbf{Brown}, et al., 2018, Gaia Data Release 2. Using Gaia parallaxes, A\&A 616, A9 
    \item Helmi, Babusiaux, Koppelman, Massari, Veljanoski, \textbf{Brown}, 2018, The merger that led to the formation of the Milky Way's inner stellar halo and thick disk, Nature 563, 7729 
\end{itemize}

\paragraph{Relevant previous projects}
Brown was the main contact point at ULEI for the following projects:
\begin{itemize}
    \item FP6 ELSA Marie-Curie Research Training Network
    \item ESF GREAT Research Network Programme 
    \item FP7 GREAT Marie-Curie Initial Training Network
    \item FP7 GENIUS Collaborative project (SPA.2013.2.1-01 Exploitation of space science and exploration data)
    \item BIG-SKY-EARTH Trans Domain COST Action
\end{itemize}

\paragraph{Significant infrastructure}
Leiden Observatory shares a building with the Lorentz Center, an international centre for workshops in the sciences which offers excellent facilities for hosting workshops, conferences, and schools. In particular, office space is available for bringing researchers together for extended periods.