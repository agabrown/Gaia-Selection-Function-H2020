\subsection{New York University}
\label{sec:nyu}

\paragraph{General description}

\paragraph{Key persons}
David W. Hogg (male) is Professor of Physics and Data Science in the Center for Cosmology and Particle Physics in the Department of Physics at New York University. He is also Group Leader for the Astronomical Data Group in the Center for Computational Astrophysics of the Flatiron Institute, a staff member in the Centre for Data Science at New York University, and he has an affiliation with the Max-Planck-Institut für Astronomie. His main research interests are in observational cosmology, especially approaches that use galaxies to infer the physical properties of the Universe. He also works on the properties and kinematics of stars in the Galaxy, and the measurement and discovery of planets around other stars. In all areas, he is interested in developing the engineering systems that make these projects possible. He has published 250 peer-reviewed publications and has supervised or co-supervised 11 PhD students and numerous postdoctoral researchers. He is the chair of the organising committee for all Gaia Sprints held to date.


\paragraph{Relevant publications}
\begin{itemize}
    \item Bonaca, Hogg, et al., 2019, The Spur and the Gap in GD-1: Dynamical Evidence for a Dark Substructure in the Milky Way Halo, ApJ, 880, 38
    \item Anderson, Hogg, et al., 2018, Improving Gaia Parallax Precision with a Data-driven Model of Stars, AJ, 156, 145
    \item Bonaca \& Hogg, 2018, The Information Content in Cold Stellar Streams, ApJ, 867, 101
    \item Eilers, Hogg, Rix, \& Ness, 2019, The Circular Velocity Curve of the Milky Way from 5 to 25 kpc, ApJ, 871, 120
    \item Rezaei Kh, Bailer-Jones, Hogg, et al., 2018, Detection of the Milky Way spiral arms in dust from 3D mapping, A\&A, 618, 168
\end{itemize}

\paragraph{Relevant previous project}
\begin{itemize}
    %NSF Astronomy and Astrophysics Research Grant AST-1517237, New Probabilistic Methods for Observational Cosmology; NSF Cyber-Enabled Discovery Type I Grant (IIS-1124794), A Unifored Probabilistic Model of Astronomical Imaging; NASA HST Archival Research Grant (AR-13250), Probabilistic Self-calibration of the WFC3 IR Channel; NSF Astronomy and Astrophysics Research Grant (AST-0908357), Dynamical models from kinematic data: The Milky Way Disk and Halo
    \item NSF Astronomy and Astrophysics Research Grant AST-1517237, New Probabilistic Methods for Observational Cosmology
    \item NSF Cyber-Enabled Discovery Type I Grant (IIS-1124794), A Unified Probabilistic Model of Astronomical Imaging
    \item NASA HST Archival Research Grant (AR-13250), Probabilistic Self-calibration of the WFC3 IR Channel
    \item NSF AST-0908357, Dynamical models from kinematic data: The Milky Way Disk and Halo
\end{itemize}

\paragraph{Significant infrastructure}
New York University and the Flatiron Institute are both located in New York City. Together these institutes host hundreds of visiting researchers every year and the Flatiron Institute has been host for multiple Gaia Sprints in the past, offering a unique and collaborative space that cultivates a lively atmosphere for research.
Hogg has strong collaborations with astrophysicists and data scientists from all research institutions in the New York area.
