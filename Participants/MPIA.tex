\subsection{Max-Planck-Institut f\"ur Astronomie (MPIA)}
\label{sec:mpg}

\paragraph{General description}
MPIA is a large and leading research institution, with three departments 'Galaxies', 'Star and Planet Formation', and (as of 1.3.2020) 'Atmosphere Physics of Exoplanets'. Its research approaches encompass observation, instrumentation development, and theoretical modelling. MPIA has played a key role in Gaia (having led the DPAC CU8 development), and Gaia data are used across the institute: MPIA researches have over 150 refereed publications involving Gaia or Gaia data, since Gaia DR1. MPIA has a long-term staff of 25 members, and about 60 post-docs and 60 PhD students.

\paragraph{Key persons} 
Hans-Walter Rix is director of the 'Galaxies and Cosmology' department at MPIA,
and adjunct faculty at Heidelberg University. He has published over 500 peer-refereed papers (about 150 of them on the Milky Way or Gaia),
and over the last 20 years has supervised 15 students, and 30 post-docs.
He has held an ERC Advanced Grant on studying the dynamical structure of the Galactic disk, and is now the project scientist for the SDSS-V survey: the only survey to provide consistent all-sky, high-quality spectra of stars to complement Gaia. 

\paragraph{Relevant publications}
\begin{itemize}
    \item Xue, \textbf{Rix} et al., 2008, ApJ, 684, 1143 (527 cit), Milky Way's Dark Matter Halo Mass
    \item Bovy \& \textbf{Rix}, 2013, ApJ, 779, 115 (290 cit) A Direct Dynamical Measurement of the Milky Way's Disk Surface Density Profile, Disk Scale Length, and Dark Matter Profile 
    \item \textbf{Rix} \& Bovy, 2013, A\&A Review, 21, 61 (165 cit.) Mapping and modelling the Galactic disk
    \item Eilers, Hogg \& \textbf{Rix}, 2019, ApJ, 871, 120E (51 cit.) The Circular Velocity Curve of the Milky Way from 5 to 25 kpc
    \item El-Badry \& \textbf{Rix} , 2019, MNRAS, 489, 5822 (13 cit.) Gaia discovery of an equal-mass `twin' binary population reaching 1000 + au separations
\end{itemize}

\paragraph{Relevant previous project}
\begin{itemize}
    \item FP7-PEOPLE-2007-1-1-ITN
    \item ERC-2012-ADG-20120216
    \item FP7-INFRASTRUCTURES-2012-1
\end{itemize}

\paragraph{Significant infrastructure}
MPIA employs $>$100 PhD and PostDoc students: for them it offers 
a rich and structured research environment, a mentoring system
and career development. MPIA has privileged access to MPG supercomputing resources,
and has extensive in-house computing. MPIA can host (also extended) workshop, from days to weeks, and from 5 to 100 persons. 
MPIA has extensive access to facilities for Gaia follow-up observations.


