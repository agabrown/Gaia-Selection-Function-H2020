\subsection{Monash University}
\label{sec:mona}

\paragraph{General description}
Monash University is a world-class teaching and research institution located in Melbourne, Australia. The School of Physics \& Astronomy at Monash University includes researchers spanning observational and theoretical astrophysics, with focussed groups on exoplanets, binary stars, stellar astrophysics, galactic structure, gravitational waves, and more. Statistical inference and data analysis are common to all of these research groups, making Monash University an excellent place to host researchers who are developing the Gaia selection function and applying it to key areas in astrophysics.

\paragraph{Key persons}
Dr.~Andrew Casey (male) is a faculty member at Monash University and an Australian Research Council DECRA Fellow. He is currently six years post-PhD and has an extensive track record of high-quality peer-reviewed publications: $>$100 publications that have accrued 5,988 citations, yielding a h-index of 34 and a m-index of 3.4. Casey is a co-organiser of the Gaia Sprints workshops (2016--2019, inclusive), which have helped strengthen the astrophysics community who are working with Gaia data.


\paragraph{Relevant publications}
\begin{itemize}
	\item \tetxbf{Casey}, et al., 2019, A Data-driven Model of Nucleosynthesis with Chemical Tagging in a Lower-dimensional Latent Space, 2019, ApJ, 887, 73
	\item \textbf{Casey}, et al., 2019, Tidal Interactions between Binary Stars Can Drive Lithium Production in Low-mass Red Giants, 2019, ApJ, 880, 125
	\item Simpson, Martell, Da Costa, \textbf{Casey}, et al., 2019, The GALAH survey: co-orbiting stars and chemical tagging, MNRAS, 482, 5302
	\item Williams, Belokurov, \textbf{Casey}, Evans, 2017, On the run: mapping the escape speed across the Galaxy with SDSS, MNRAS, 468, 2359
	\item \textbf{Casey} et al., 2017, The RAVE-on Catalog of Stellar Atmospheric Parameters and Chemical Abundances for Chemo-dynamic Studies in the Gaia Era, ApJ, 840, 59
\end{itemize}

\paragraph{Relevant previous projects}

\begin{itemize}
    \item Gaia-ESO survey. A large European survey which was specfically aimed at collecting spectrocopic data complementary to the Gaia mission.
\end{itemize}

\paragraph{Significant infrastructure}
As of 2020 Monash University joins the Australian Research Council Centre of Excellence in Astrophysics in 3-Dimensions (ASTRO 3D), a \$30\,M AUD (18.3\,M Euro) initiative for astrophysics research in Australia. This initiative includes researchers working on the formation and evolution of the Milky Way galaxy, single and binary star evolution, and other key science goals for the Gaia mission. The Australian-led GALAH survey is one of the key projects of ASTRO 3D, with the goal to obtain high-resolution optical spectra for up to $\sim10^6$ stars. Currently the GALAH/Gaia sub-sample represents the highest quality set of stars with precise astrometry and chemical abundance information. This provides a unique opportunity to study the chemo-dynamical history of the Milky Way in exquisite detail, yet the Gaia and GALAH samples are both subject to selection function effects. Monash University is the ideal place for researching binary population inference with the Gaia selection function, and the complement of the Gaia selection functions with other surveys (e.g. GALAH).
