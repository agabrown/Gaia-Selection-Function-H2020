\subsection{INAF - Osservatorio Astrofisico di Torino}
\label{sec:inaf}

\paragraph{General description}

The Istituto Nazionale di Astrofisica (INAF) promotes, coordinates, and carries out, within the context of programs of the European Union and other International Entities, research activities in the fields of Astronomy, Radioastronomy, Space Astrophysics, and Cosmic Physics, in collaboration with Italian Universities as well as other private and public, national and international research organizations. In particular, INAF coordinates research activities carried out at national facilities (the Telescopio Nazionale Galileo, TNG, in the Canary Islands, and the Large Binocular Telescope, LBT, on Mt. Graham, Arizona) and at 18 research institutes and astronomical observatories on Italian soil. The proposed project will be undertaken at the Osservatorio Astrofisico di Torino (OATo), one of the INAF institutes. 

\paragraph{Key persons}

Dr.~Ronald Drimmel (male) is a research astronomer (staff) of INAF whose research interest is primarily the structure and dynamics of the Milky Way, especially the investigation of non-axisymmetric structures such as the spiral arms and the warp. While currently he works with Gaia data, he has analysed infrared and far-infrared all-sky data from the COBE satellite to derive a model of the distribution of the stars and interstellar dust in the Milky Way on large scales. Drimmel has been a key member of the Gaia Data Processing and Analysis Consortium (DPAC), serving as it's Deputy Chair from 2006 to 2012. Currently he is contributing to the construction of an all-sky map of the total Galactic extinction. Drimmel is also on the Gaia Payload Experts group, responsible for monitoring and evaluating the performance of the Gaia instrument.

Dr.~Richard Smart (male) has been active in Gaia since it's conception. He was the provider of the Initial Gaia Source List \cite{2014A&A...570A..87S} and is member of the Gaia Ultra Cool Dwarf and Gaia Ground Based Tracking work package and is leading the effort to produce the Gaia Catalogue of Nearby Stars that contains all stars to M9 within 100~pc of the Sun. Smart was the coordinator of the EU FP7 RISE program Interpretation and Parameterization of Extremely COOL objects (IPERCOOL, 247593), the Italian coordinator for the GENIUS  program and has managed 5 postdocs at the Observatory. 

Dr.~Alessandro Spagna (male) is a senior research astronomer (staff) of INAF. His main scientific interests are in the field of ground-based and space astrometry, in the construction of wide field surveys for the study of the Galactic stellar populations and the chemo-dynamical evolution of the Milky Way. In particular his research activity focuses on the formation of the Galactic thick disc and stellar halo. He has been also involved in various projects for the construction of all-sky catalogues, such as Hipparcos, Guide Star Catalog II, and Gaia. 

Dr. Eloisa Poggio (female) is a postdoctoral fellow at INAF and member of DPAC. Her main scientific interests include the structure and dynamics of the disk of the Milky Way, with particular focus on non-axisymmetric features such as the warp and spiral arms. She is currently developing a model for the Galaxy, aimed at estimating Galactic parameters via statistical bayesian inference including the survey selection function.



\paragraph{Relevant publications}
\begin{itemize}
    \item Poggio, \textbf{Drimmel}, et. al., 2020, Evidence of a dynamically evolving Galactic warp, Nature Astronomy (doi:10.1038/s41550-020-1017-3)
    \item Poggio, \textbf{Drimmel}, et. al., 2018, The Galactic warp revealed by Gaia DR2 kinematics, MNRAS Letters 481, L21
    \item  Gaia Collaboration, Katz, Antoja, Romero-G\'omez, \textbf{Drimmel}, et al., 2018 Gaia Data Release 2. Mapping the Milky Way disc kinematics A\&A 616, A11
    \item \textbf{Drimmel} \& Spergel, 2001, Three-dimensional Structure of the Milky Way Disk: The Distribution of Stars and Dust beyond 0.35~$R_\mathrm{solar}$ ApJ, 556, 181
    \item \textbf{Drimmel}, 2000, Evidence for a two-armed spiral in the Milky Way A\&A, 358, L13
\end{itemize}

\paragraph{Relevant previous projects}

Drimmel was the main contact point at INAF for the following past projects:
\begin{itemize}
    \item FP6 ELSA Marie-Curie Research Training Network
\end{itemize}
Drimmel also participated in the following EU funded projects: 
\begin{itemize}
    \item ESF GREAT Research Network Programme 
    \item FP7 GREAT Marie-Curie Initial Training Network
    \item FP7 GENIUS Collaborative project (SPA.2013.2.1-01 Exploitation of space science and exploration data)
\end{itemize}


\paragraph{Significant infrastructure}

In the context of this proposal, the most relevant facilities that INAF participants have access to is the Marconi supercomputer at the CINECA, a national high-performance computing centre. Marconi is classified in Top500 list among the most powerful supercomputer:  rank 12 in November 2016, and rank 19 in the November 2019 list. Also available is an Oracle Database Machine at Trieste Observatory.