\subsection{INAF - Osservatorio Astrofisico di Torino}
\label{sec:inaf}

\paragraph{General description}

The Istituto Nazionale di Astrofisica (INAF) promotes, coordinates, and carries out, within the context of programs of the European Union and other International Entities, research activities in the fields of Astronomy, Radioastronomy, Space Astrophysics, and Cosmic Physics, in collaboration with Italian Universities as well as other private and public, national and international research organizations. In particular, INAF coordinates research activities carried out at national facilities (the Telescopio Nazionale Galileo, TNG, in the Canary Islands, and the Large Binocular Telescope, LBT, on Mt. Graham, Arizona) and at 18 research institutes and astronomical observatories on Italian soil. The proposed project will be undertaken at the Osservatorio Astrofisico di Torino (OATo), one of the INAF institutes. 

\paragraph{Key persons}

Ronald Drimmel is a research astronomer (staff) of INAF whose research interest is primarily the structure and dynamics of the Milky Way, especially the investigation of non-axisymmetric structures such as the spiral arms and the warp. While currently he works with Gaia data, he has analysed infrared and far-infrared all-sky data from the COBE satellite to derive a model of the distribution of the stars and interstellar dust in the Milky Way on large scales. 

Ronald Drimmel has been a key member of the Gaia Data Processing and Analysis Consortium (DPAC), serving as it's Deputy Chair from 2006 to 2012. Currently Ron is contributing to the construction of an all-sky map of the total Galactic extinction. Ron is also on the Gaia Payload Experts group, responsible for monitoring and evaluating the performance of the Gaia instrument.

\paragraph{Relevant publications}
\begin{itemize}
    \item Poggio, Drimmel, et. al., 2020, 'Evidence of a dynamically evolving Galactic warp', Nature Astronomy (doi:10.1038/s41550-020-1017-3)
    \item Poggio, Drimmel, et. al., 2018, 'The Galactic warp revealed by Gaia DR2 kinematics', MNRAS Letters 481, L21
    \item  Gaia Collaboration, Katz, Antoja, Romero-G\'omez, Drimmel, et al., 2018 'Gaia Data Release 2. Mapping the Milky Way disc kinematics' A\&A 616, A11
    \item Drimmel and Spergel, 2001, 'Three-dimensional Structure of the Milky Way Disk: The Distribution of Stars and Dust beyond 0.35 Rsolar' ApJ, 556, 181
    \item Drimmel, 2000, 'Evidence for a two-armed spiral in the Milky Way' A\&A, 358, L13
\end{itemize}

\paragraph{Relevant previous projects}

Drimmel was the main contact point at INAF for the following past project:
\begin{itemize}
    \item FP6 ELSA Marie-Curie Research Training Network
\end{itemize}
Drimmel also participated in the following EU funded projects: 
\begin{itemize}
    \item ESF GREAT Research Network Programme 
    \item FP7 GREAT Marie-Curie Initial Training Network
    \item FP7 GENIUS Collaborative project (SPA.2013.2.1-01 Exploitation of space science and exploration data)
\end{itemize}


\paragraph{Significant infrastructure}