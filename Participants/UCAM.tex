\subsection{Institute of Astronomy at the University of Cambridge (UCAM)}
\label{sec:ucam}

\paragraph{General description}

Institute of Astronomy (IoA) is one of the three Astronomy departments at the University of Cambridge, the other two being Cavendish Astrophysics Group and the Astrophysics Group at the Department of Applied Mathematics and Theoretical Physics. The IoA has 18 permanent staff members, some 100 postdoctoral fellows, approximately 50 graduate students and 25 support staff. The Institute also hosts the Cambridge Astronomical Survey Unit, the Gaia’s Coordination Unit 5 in charge of Photometric Processing and the Gaia-ESO survey. The Institute’s research portfolio includes the full spectrum of the cutting edge Astrophysics topics, from the Early Universe to Exo-planets and instrument design and construction. The Institute played a uniquely active role in the scientific exploration of the Gaia DR1 and DR2 resulting in more than 200 publications by the IoA members (over the last 4 years only). The direct daily interface between the broad and active research body and CASU and CU5 makes UCAM an especially attractive environment for the Proposed Project.


\paragraph{Key persons}

Prof.~Vasily Belokurov (male) is a Professor of Astronomy at the University of Cambridge. He is a survey scientist, whose expertise is in data mining, galactic dynamics and machine learning. The broad theme of his research is galaxy formation and evolution, with a particular focus on the Milky Way halo, low-mass galaxies and local constraints on Dark Matter. Belokurov has supervised more than 20 graduate students and worked closely with an appreciable number of postdocs. He has published more than 250 papers in refereed journals. It is worth mentioning separately the papers that present the analysis based on the data from the ESA’s Gaia mission. Gaia Data Release 1 was delivered in 2016, while the Gaia DR2 arrived less than 2 year ago. So far, Belokurov has published $>$40 papers in refereed journals, relying on either Gaia DR1 or DR2 data.

Andrew Everall (male) is a PhD student at the Institute of Astronomy working with supervisors Prof.~Vasily Belokurov and Prof.~Wyn Evans. His core focus is on constraining the Milky Way Dark Matter distribution and specifically the local DM density. He has worked significantly on selection functions for multi fibre spectrographs and has already performed a substantial amount of work on selection functions for Gaia DR2 and subsets therein with Dr.~Douglas Boubert. 

\paragraph{Relevant publications}

\begin{itemize}
    \item Boubert \& \textbf{Everall}, submitted, Completeness of the \textit{Gaia}-verse II: what are the odds that \textit{Gaia} DR2 missed a star?
    \item Boubert, \textbf{Everall} \& Holl, submitted, Completeness of the Gaia-verse I: when was Gaia’s eye off the sky?
    \item Grady, \textbf{Belokurov} \& Evans, 2020, MNRAS, 492, 3128, Age demographics of the Milky Way disc and bulge
    \item Deason, \textbf{Belokurov} \& Sanders, 2020, MNRAS, 490, 3426, The total stellar halo mass of the Milky Way
    \item \textbf{Everall} \& Das, 2020, MNRAS, 493, 2042, seestar: Selection functions for spectroscopic surveys of the Milky Way
    \item \textbf{Everall}, Evans, \textbf{Belokurov}, Schönrich, 2019, MNRAS, 489, 910, The tilt of the local velocity ellipsoid as seen by Gaia
    \item Erkal, \textbf{Belokurov}, et al., 2019, MNRAS, 487, 2685, The total mass of the Large Magellanic Cloud from its perturbation on the Orphan stream
    \item Torrealba, \textbf{Belokurov}, et al., 2019, MNRAS, 488, 2743, The hidden giant: discovery of an enormous Galactic dwarf satellite in Gaia DR2
    \item Necib, Lisanti, \textbf{Belokurov}, 2019, ApJ, 874, 3
    \item \textbf{Belokurov}, Erkal, 2019, MNRAS, 482, L9
    \item \textbf{Belokurov} et al., 2018, MNRAS, 478, 611, Co-formation of the disc and the stellar halo
\end{itemize}

\paragraph{Relevant previous projects}

\begin{itemize}
    \item Starting Grant (StG), PE9, ERC-2012-StG\_20111012
    \item Gaia-ESO survey. A large European survey which was specfically aimed at collecting spectrocopic data complementary to the Gaia mission.
\end{itemize}

\paragraph{Significant infrastructure}